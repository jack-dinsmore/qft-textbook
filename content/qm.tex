\chapter{Review of Non-relativistic Quantum Mechanics}

\section{Tenets of Quantum Mechanics}

\section{\Schrodinger and Heisenberg Picture}

\section{Non-Relativistic Scattering}

Since the predominant use of relativistic QFT is to compute scattering cross sections, it is worth reviewing non-relativistic scattering in quantum mechanics. Consider a particle propelled toward some potential $V(\bm x)$ which tails off at large $\bm x$. For example, an electron incident on a proton with Coulomb attraction. When the particle is far from the potential, its wavefunction is a superposition of plane waves. For simplicity, suppose the initial state is just one plane wave of momentum $\bm p$:
\begin{e}
  \ket{\psi(t\rightarrow -\infty)} = \ket {\bm p} = \int dx\, e^{i\bm p \cdot x}\ket x.
  \label{eqn:non-rel-scattering-initial}
\end{e}
The final state is also far from the potential, so it is a superposition of many plane waves
\begin{e}
  \ket{\psi(t\rightarrow +\infty)} = A \ket {\bm p} - i\int d^3 \bm k\, \ket {\bm k} M(\bm k).
  \label{eqn:non-rel-scattering-final}
\end{e}
To understand what $M(\bm p)$ is, use the fact that $\braket{\bm k|\bm p} = \delta(\bm k - \bm p)$ so that
\begin{e}
  \braket{\bm k | t\rightarrow +\infty} = A\delta(\bm k - \bm p) - iM(\bm k).
  \label{eqn:non-rel-scattering-m-def}
\end{e}
Thus, $|A|^2$ is the probability for the particle not to be scattered at all, and $|M(\bm k)|^2$ is the probability of scattering into momentum $\bm k$.

Our goal is to compute $M(\bm k)$ using the \Schrodinger equation, which states that (for a static potential)
\begin{e}
  \ket{\psi(t)} = e^{-2iHt}\ket{\psi(-t)}.
  \label{eqn:non-rel-scattering-schrodinger}
\end{e}

Since our initial and final states are simple in the momentum basis, we'll write $H$ in the momentum basis too. The kinetic energy part of $H$ is simple:
\begin{e}
  K = \int d^3 \bm k\, \ket{\bm k}\frac{k^2}{2m}\bra{\bm k}
\end{e}
and the potential energy can be written in the momentum basis via its Fourier transform $V(\bm k)$ and the fact that $\braket{p|x} = e^{i\bm k\cdot x}$.
\begin{es}
  V &= \int d^3 \bm x\, \ket{\bm x} V(\bm x) \bra{\bm x} \\
  &= \int d^3 \bm x\, d^3 \bm k\, d^3 \bm k'\, \ket{\bm k}\braket{\bm k | \bm x}V(\bm x) \braket{\bm x|\bm k'}\bra{\bm k'}\\
  &= \int d^3 \bm x\, d^3 \bm k\, d^3 \bm k'\, \ket{\bm k}e^{-i(\bm k'-\bm k) \cdot \bm x} V(\bm x) \bra{\bm k'}\\
  &= \int d^3 \bm k\, d^3 \bm k'\, \ket{\bm k} V(\bm k' - \bm k) \bra{\bm k'}.\\
\end{es}

We'll consider the case of weak scattering, where $V$ is usually small\footnote{For example, the expectation value of the scattered particle's kinetic energy is much greater than its potential energy.} In this case, the particle will spend only a short time in the potential well, so we can strategically pick the time $t$ such that $\ket{\psi(t)}\approx \ket{\psi(\infty)}$ (and $\ket{\psi(-t)}\approx \ket{\psi(-\infty)}$), but $t$ is still small. We will therefore drop all terms of order $\mathcal{O}(t^3)$, $\mathcal{O}(t^2 V)$, and $\mathcal{O}(tV^2)$ in our calculation for $M$:
\begin{es}
  \braket{\bm k | \psi(t)} &= \braket{\bm k | \brackets{1 - 2iKt - 2 K^2 t^2 - 2i V t} | \psi(-t)}\\
  &= \parens{1 - i \frac{k^2}{m} - \frac{k^4}{2m^2}}\braket{\bm k|\bm p} - 2it\int d^3 k'\, V(\bm k' - \bm k)\braket{\bm k'|\bm p}.\\
\end{es}
Equation (\ref{eqn:non-rel-scattering-m-def}) connects the left hand side to the scattering amplitude $M$. Since $\braket{\bm k|\bm p}$ is the same $\delta(\bm k - \bm p)$ as the coefficient of $A$, we identify $A$ with the first term of the previous equation and $iM$ with the second:
\begin{ec}
  A = 1 - i t\frac{p^2}{m} - t^2\frac{p^4}{2m^2}\\
  M(\bm k) = 2tV(\bm k - \bm p).
\end{ec}

This is an interesting result: $|A|^2 = 1$ to order $t^2$, so that an unscattered particle is offset in phase by $\Delta \phi = -tp^2/m$. \jtd{Somehow this phase is -1}. It follows that our scattering amplitude is
\begin{e}
  M(\bm k) = 2\frac{m}{p^2}V(\bm k - \bm p).
\end{e}

For an inelastic position (i.e., when the potential contains no internal degrees of freedom which could steal energy form the particle), we expect the initial and final energies to be the same. This implies that $M(\bm k) \propto \delta(k^2 - p^2)$ so that the scattering amplitude lies on a sphere of radius $p^2$.

\jtd{Alternate method}

This time, instead of framing the problem as a single particle moving from late times to early times, we'll imagine a stream of particles continuously scattering from a distant source. At far distances, the wavefunction is a superpostion of the incoming plane wave and the scattered wavefunction
\begin{e}
  \psi(\bm r) = e^{i \bm k \cdot \bm x} + \phi(\bm r),
\end{e}
and this wavefunction satisfies the \Schrodinger equation
\begin{e}
  H \psi(\bm r) = E \psi(\bm r).
\end{e}
Splitting the Hamiltonian into kinetic and potential energy $H = K+V$,
\begin{e}
  (E-K) \psi(\bm r) = V \psi(\bm r).
\end{e}
Suppose we have some function $G(\bm x)$, which satisfies
\begin{e}
  (E-K) G(\bm x) = \delta(\bm x).
\end{e}
This equation is useful because it suggests that the following fact is true of $\psi(\bm x)$
\begin{e}
  \psi(\bm r) - \psi_0(\bm r) = \int d^3\bm x'\, G(\bm x-\bm x')V(\bm x')\psi(\bm r)
  \label{eqn:non-rel-full-scattering}
\end{e}
where $\psi_0(\bm r)$ is the wavefunction of free space, where $V = 0$. \jtd{Explain the fact that if we use an advanced Green's function, we can use $\psi_0 =$ plane wave.}

The left hand side of (\ref{eqn:non-rel-full-scattering}) is therefore the scattered wavefunction $\phi(\bm r)$. A simple yet powerful approximation is to set $\phi(\bm r)$ in the integrand equal to $\psi_0(\bm r)$. This is known as the \emphi{Born approximation}\index{Born approximation}, and it readily gives the scattered wavefunction.



% In a moment, we will take $t\rightarrow \infty$ to recover the initial and final states. Let us suppose that the particle is moving fast enough that it spends little time in the potential --- that is, $t$ is a small parameter. Then we can expand the exponent to linear order and dot with $\bra{\bm k}$
% \begin{e}
%   \Braket{\bm k | \psi(t)} = \Braket{\bm k |(1 + 2iHt)| \psi(-t)}.
% \end{e}

% Writing the Hamiltonian as $H = K + V$ where $K$ is kinetic energy satisfying $\bra{\bm k}K = \frac{k^2}{2m}\bra{\bm k}$, the above simplifies to
% \begin{e}
%   \Braket{\bm k | \psi(t)} = \braket{\bm k |\psi(-t)}\parens{1 + it\frac{k^2}{m}} + 2it \braket{\bm k |V|\psi(-t)}.
%   \label{eqn:non-rel-scattering-v}
% \end{e}

% So far we have remained agnostic of this potential $V$, but imagine we have derived its Fourier transform. That is, we know $V(\bm k)$ such that 
% \begin{e}
%   V = \int \frac{d^3 \bm k}{(2\pi)^3} \ket{\bm k}V(\bm k)\bra{\bm k}.
% \end{e}
% Then (\ref{eqn:non-rel-scattering-v}) gives us
% \begin{e}
%   \Braket{\bm k | \psi(t)} = \braket{\bm k |\psi(-t)}\parens{1 + it\frac{k^2}{m} + \frac{it}{4\pi^2} V(\bm k)}.
% \end{e}

% Sending $t\rightarrow -\infty$ allows us to use (\ref{eqn:non-rel-scattering-initial}--\ref{eqn:non-rel-scattering-m-def}) to simplify $\braket{\bm k | \psi}$:
% \begin{e}
%   iM(\bm k) = \delta(\bm k-\bm p)\parens{1 + it\frac{k^2}{m} + \frac{it}{4\pi^2} V(\bm k)}.
% \end{e}


% Equation \jtd{blank} gives us an excellent intuition for how an scattering experiment measures of the non-relativistic notion of a potential. We will routinely use it to understand our predictions for QFT scattering experiments later in the book. The approach we used to derive the formula is also useful, since we will use a similar experiment to derive QFT cross sections, known as $S$-matrices.