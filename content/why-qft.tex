\chapter{Principles of Quantum Field Theory}
\label{chap:intro}

\subsection{Historical Introduction}

The first three decades of the 20th century marked an unparalleled advance in fundamental physics. Sparked by the observation of black body radiation --- specifically its exponential frequency cutoff which suppresses high-energy radiation --- Max Planck postulated in 1900 that the energy of light had to come in \emphi{quanta}. Inspired, Albert Einstein used this idea of light quanta to explain the photoelectric effect, and Niels Bohr and Ernest Rutherford generalized the idea of quanta by suggesting in 1913 that angular momentum was also quantized in order to explain atomic emission lines. With Louis de Broglie's proposal in 1924 that matter was also quantized, it became rare to find properties of nature that were \textit{not} quantized --- a statement made rigorous by the discovery of the \Schrodinger equation in 1926, and the foundation of quantum mechanics.

Independently, Maxwell's prediction and the following experimental confirmation that the speed of light be constant had initiated its own cascade of discoveries. In 1905, Einstein reshaped our definitions space and time by uniting them into one entity --- spacetime --- in order to apply Maxwell's equations of electromagnetism in moving reference frames. His new theory is now called special relativity. Ten years later, he unified special relativity with gravity to produce general relativity, which explained the slow precession of Mercury's orbit, and now explains much more exotic phenomena such as gravitational lensing, black holes, and the history of the universe.

Though these two theories of relativity and quantum mechanics were each incredibly successful, they were completely in conflict. The famous \Schrodinger equation, though in agreement with experiment, contains one time derivative and two position derivatives:\footnote{This is the first time we have used natural units, wherein $\hbar = c = 1$.}
\begin{e}
  i\partial_t \psi = \parens{-\frac{1}{2m}\partial_x^2}\psi.
  \label{eqn:shrodinger}
\end{e}
We've set potential energy equal to zero assuming that the particle is in vacuum. According to special relativity, a Lorentz boost to a new reference frame with velocity $v$ must introduce the substitutions $\del_t \rightarrow \del_t + v \del_x$ and $\del_x \rightarrow \del_x + v \del_t$. But that would introduce additional space and time derivatives so that the result is no longer the \Schrodinger equation. Thus, (\ref{eqn:shrodinger}) equation is inconsistent with relativity.

A few experimental anomalies had also begun to appear by the late 1920s. It was becoming increasingly apparent that electron spin, largely pioneered by Pauli, needed to be introduced to quantum mechanics. The spectrum of hydrogen, once the poster child of the quantization of angular momentum, was showing a small discrepancy with the theoretical prediction of $s$ orbital energy. Separately, Compton scattering experiments where high energy photons were scattered off electrons showed that light had important quantum mechanical properties which could not be understood with the \Schrodinger equation.

Paul Dirac thought that the solution to some of these problems might be the introduction of relativity to quantum mechanics. There were two examples of equations of motion already introduced to physics that were consistent with relativity: one was Maxwell's equations (in vacuum), written in terms of the vector potential $A^\mu$
\begin{e}
  \del_\mu \parens{\del^\mu A^\nu - \del^\nu A^\mu} = 0
  \label{eqn:maxwell}
\end{e}
and the other was the wave equation, also called the Klein-Gordon equation, written for a wavefunction $\phi$:
\begin{e}
  \del^\mu \del_\mu \phi^2 - m\phi^2 = 0.
  \label{eqn:klein-gordon}
\end{e}
Both of these equations are Lorentz invariant, though they are not good replacements for the \Schrodinger equation since they do not have the same non-relativistic limit. For the same reason, they also don't resolve the new experimental anomalies. Instead, Dirac created a new relativistic equation. Like the \Schrodinger equation, the Dirac equation is contains one derivative and consequentially reproduces the \Schrodinger equation in the non-relativistic limit. In order to contract away the index of this single derivative $\del_\mu$, the Pauli matrices $\sigma_1$, $\sigma_2$, and $\sigma_3$ had to be combined with the identity matrix into a four-vector $\sigma^\mu = (\mathds{1}, \sigma_1, \sigma_2, \sigma_3)$ which represents the spin operator.\footnote{(\ref{eqn:dirac}) only applies to massless particles, which is why $m$ is not present in the equation. It can be generalized to apply to massive particles such as electrons with some extra mathematical machinery, which is done in chapter \ref{chap:spin-one-half}.

The fact that spin and Lorentz invariance were resolved with the same blow in (\ref{eqn:dirac}) is no accident; spin, also known as intrinsic angular momentum, is only a valid particle property because it is conserved, and angular momentum is only conserved because physics is invariant under rotations. Lorentz invariance implies invariance under rotations, so any theory which is Lorentz invariance should also come with a complementary interpretation of conserved angular momentum, or spin.}
\begin{e}
  i\parens{\del_\mu \sigma^\mu} \psi = 0.
  \label{eqn:dirac}
\end{e}
This was the first fundamental theory to describe particle spin. It also allowed the anomalous energy of the $s$ orbitals to be understood as a relativistic correction (the Darwin term), and it predicted Compton scattering cross-sections, with its relativistic electrons.

However, the Dirac equation had one glaring flaw. It predicted the existence of a new particle of equal mass to the electron but with opposite charge --- a particle which somehow had never been seen despite the recent surge of particle physics experiments. Dirac wasn't so bold as to predict a new particle without experimental evidence, so he postulated that this particle was the proton, even though experiment had already shown that the proton was much more massive than the electron.

Four years later, it appeared Dirac should have stuck to his guns. The positron was discovered by Carl David Anderson and confirmed to have the same mass and opposite charge of the electron. The new particles didn't stop there; the neutron was discovered in the same year, and muon not long after in 1936. Together with these new particles was the ever-growing theory of nuclear decay, describing how protons, neutrons, and electrons conspired to turn into each other with fixed rates. No theory yet written allowed particles to change identity or split in such a way.

In order model decay, many physicists sought to embrace relativity more fully than Dirac's equation did by fundamentally changing our understanding of particles. Non-relativistic quantum mechanics models particles as described wavefunctions, which are in turn ruled by the \Schrodinger equation or Dirac equation. But wavefunctions, which are defined as a function valid at all positions for one unique time $t$, are not a Lorentz-invariant concept. Fields, such as the electric or magnetic fields, are Lorentz invariant because they are defined for a unique spacetime coordinate. Thus, the search for ``Quantum Field Theory,'' or QFT, began. Initially, progress was slowed by the temperamental behavior of energy; the vacuum energy predicted by the \Schrodinger became infinite in a relativistic version, as did other quantities such as particle masses. Some physicists began to suggest abandoning the new QFT framework altogether, but the occasional success kept many on the QFT track. For example, a new anomaly of the hydrogen $s$ orbital known as the Lamb shift was discovered in 1947 and explained by Hans Bethe using QFT, and a correction to the magnetic moment of the electron was derived in 1948 by Julian Schwinger and experimentally confirmed.

The theoretical troubles of QFT were redressed over the next decade through a number of procedures largely developed by Julian Schwinger, Richard Feynman, and Shinichiro Tomonaga, for which they one the 1965 Nobel Prize in Physics. Their theory, named Quantum Electrodynamics (QED), explains all the phenomena of classical mechanics, non-relativistic quantum mechanics, and special relativity, as well as confirming the new discoveries of the positron, the Lamb shift, and the electron's anomalous magnetic moment. As of the 2000s, its predictions have been confirmed to as many as 12 decimal places in multiple experiments. The theory also addressed several philosophical questions, such as the origin of particle spin and the definitions of fermions and bosons. The first part of this book is dedicated to understanding this incredibly successful theory and how to reproduce its major predictions.

\subsection{The Principle of Least Action}

The three fundamental theories before QFT (classical mechanics, relativity, and quantum mechanics) can all be framed as principle of least action theories. It seems reasonable to pursue such a principle for QFT.

In classical mechanics, action $S$ is defined as the integral of some function $L$ with dimensions of energy, known as the Lagrangian, with respect to time. The integral is computed over the path $C$ taken by a particle:
\begin{e}
  S = \int_C dt \, L.
  \label{eqn:classical-action}
\end{e}
We propose as a physical law that Nature choses the path $C$ such that the action is minimized. Thus, any small change to $C$ does not lower the action:
\begin{e}
  \delta S = 0.
  \label{eqn:least-action}
\end{e}
(\ref{eqn:least-action}) is known as the principle of least action. This principle of least action can be shown to reduce to Newton's second law $F=ma$.

The advantages of the action formulation are many. Firstly, it reduces all the properties of a physical theory to one variable: the Lagrangian. All a physicist need do to create a theory is write down a new form of $L$. Furthermore, writing a new Lagrangian is made easier by the fact that all the symmetries of $S$ are symmetries of the theory in general. For example, we have observed that physics is invariant under translation --- that is, that experiments done in different locations yield the same results. Thus, $S$ cannot depend explicitly on position. Other examples of symmetry are invariance under orientation and time translation. Relativity introduces Lorentz boosting as a symmetry, while QFT will introduce even more. All of these symmetries cut down so much on the number of forms $S$ can take that Lagrangians tend to be quite simple in practice. In classical mechanics, the Lagrangian is
\begin{e}
  L_\mathrm{Classical} = K - V
  \label{eqn:classical-lagrangian}
\end{e}
where $K$ and $V$ are kinetic and potential energy. All other terms are forbidden because they do not have the right units or the right symmetries.

In general relativity, the purpose of the Lagrangian is to dictate how the metric of spacetime $g_{\mu \nu}(x)$ interacts with matter. This requires us to change our thinking, since $g_{\mu \nu}(x)$ is a field and therefore fundamentally different from a particle path $\bm x(t)$ that one might solve for in classical mechanics. In the path case, $\bm x$ is a dynamical variable indicating the one location where the particle can be. In the field case, $x$ is instead a label indicating which part of the field we're solving for; the dynamical variable is $g_{\mu \nu}$.

This distinction fuels a change in our definition of action. Instead of an integral over a path, we add up the individual Lagrangians of every $g_{\mu \nu}$ value in spacetime:
\begin{e}
  S = \int d^4 x\, \sqrt{-g}\mathcal{L}(x)
\end{e}
where $\mathcal{L}(x)$ is called the \emphi{Lagrangian density}\footnote{The $\sqrt{-g}$ term is included in the volume element because to account for the possibility that the spacetime shrinks or expands in some regions. $g$ represents the determinant of the metric.} In the name of causality, we also introduce the principle of \emphi{locality}, which means that the physics of one point in spacetime cannot be instantaneously affected by a that of a distant point. Since $\mathcal{L}(x)$ encapsulates the physics at a point, it must only depend on $x$ and derivatives with respect to $x$ rather than some distant point $y$. This principle of locality means that information can only communicated across spacetime by means of waves, such as the electromagnetic waves of Maxwell's equations, which must move at the speed of light.

Putting together the notion of a Lagrangian density with the requirement of Lorentz invariance and the principle of locality, the general relativity Lagrangian density is
\begin{e}
  \mathcal{L}_\mathrm{GR} = \sqrt{-g} \parens{R - 2 \Lambda} + \mathcal{L}_\mathrm{Matter}.
\end{e}
Here, $R$ is the Ricci scalar (the simplest non-constant Lorentz-invariant scalar defined from the metric), and $\mathcal{L}_\mathrm{Matter}$ is the Lagrangian density of matter that happens to be present. $\Lambda$ is known as the cosmological constant which roughly represents the energy of the vacuum.

Since $g_{\mu \nu}$ is a field, the above Lagrangian is our first \emphi{field theory}. Another field theory is electromagnetism, whose Lagrangian is
\begin{e}
  \mathcal{L}_\mathrm{EM} = -\frac{1}{4}F_{\mu \nu} F^{\mu \nu}
\end{e}
where $F^{\mu \nu}$ is the Faraday tensor $F^{\mu \nu} = \del^\mu A^\nu - \del^\nu A^\mu$.
These field theories are still classical because they do not produce quantized particles., so covering the principle of least action for a quantum theory is the next step.

The success of quantum mechanics is largely due to the presence of the new constant $\hbar$, which has units of energy times time. $\hbar$ is necessary to quantize energy, angular momentum, and to write the \Schrodinger equation, but $\hbar$ poses a problem for the principle of least action. Action has the same units as $\hbar$, so the principle of least action is less justified. Why can't $\delta S = \hbar$? Or $\delta S = 2\hbar$? Furthermore, how do we deal with the fact that quantum particles have wavefunctions and don't exist at well-defined points?

Paul Dirac developed a simple solution to both these problems in 1933, which was later formalized by Richard Feynman. To compute the probability that a quantum particle could go from position $\ket{x(0)}$ at time $t=0$ to position $\ket{y(t)}$ at time $t$, one need only calculate
\begin{e}
  \braket{\bm x(t) | \bm y(0)} \propto \sum_{\mathrm{paths}} e^{-i S / \hbar}.
  \label{eqn:quantum-least-action}
\end{e}
and the probability is merely $|\braket{\bm x(t)\bm y(0)}|^2$. Here, $S$ is the classical action (\ref{eqn:classical-action}) of the Lagrangian integrated over a path, and the sum indicates a sum over all possible paths between point $\bm y$ at time $t=0$ to point $\bm x$ at time $t$. The proportionality merely indicates that this equation does not automatically normalize the probability; the user has to do this after computing $\braket{\bm x(t) | \bm y(0)}$.

Surprisingly, (\ref{eqn:quantum-least-action}) is almost equivalent to the classical principle of least action $\delta S = 0$. Consider a the classical path $C$, which contributes a term of $e^{iS}$ to $\braket{x(0)|y(t)}$, where $S$ is the classical action. A nearby path will have almost the same action, since by definition $\delta S = 0$ for a classical path When added, $e^{iS}$ values for both paths will have almost the same phase and will constructively interfere. But if we consider a path not close to the classical path, its neighbors will have dissimilar actions and will destructively interfere. Thus, the paths that contribute the most to $\braket{x(0)|y(t)}$ are those that lie near the classical path.\footnote{The rigorous version of this statement is known as the ``stationary phase approximation.''}

It is also easy to see how this definition might give rise to some quantum properties. If one shines an electron beam at two parallel slits, each electron may take one of two classical paths. The interference between the $e^{iS}$ values of these paths corresponds exactly to the interference pattern made by the electron on a screen behind the slits. To give another example, when computing decay rates of a quantum system such as an atom, one finds that the way to compute a decay rate is to add up the rate of every possible decay path equally, just as the magnitude $|e^{iS}|=1$ is equal for all paths.

\subsection{Quantum Field Theory Lagrangian}
As discussed in the abstract, the best way to guarantee a relativistic quantum theory is do away with the concept of wavefunctions and instead use fields, which automatically ensure that information is not transferred faster than the speed of light. So we create a field $\hat \phi(x)$ which is a quantum operator. It is defined so that the state $\phi(x)\ket{0}$ represents a single particle state with a particle existing at position $x$. In this field language, the $\braket{\bm x(t)|\bm y(0)}$ in (\ref{eqn:quantum-least-action}) is written as $\braket{0|\hat\phi(x)^\dagger \hat\phi(y)|0}$ where $x = (t,\bm x)$ and $y=(0, \bm y)$ are four-vectors.

To write an action for the operator $\hat \phi(x)$, we'll borrow the field theory Lagrangian of general relativity:
\begin{equation}
  S = \int d^4\, \mathcal{L}(x)
\end{equation}
where we have dropped the $\sqrt{-g}$ term because we will always work in Minkowski space in this book, where $g= -1$.

The last step is to write a principle of least action and we'll borrow the quantum idea of adding contributions from all paths. However, in the non-relativistic quantum case $\bm x$ was a dynamical variable, so the relevant paths to sum over were paths through space --- that is, through values of $\bm x$. Now the dynamical variable is the value of a field, so the path should be all the values that the field can take at all points. We'll define a shorthand for summing over these paths
\begin{e}
  \int_\mathrm{paths}\phi(\mathrm{all}\ x) = \prod_{x\ \in\ \mathrm{spacetime}}\int \phi(x) \equiv \int \mathcal{D}\phi
\end{e}
and use it to make our first QFT principle of least action:
\begin{e}
  \braket{0|\hat\phi(x)^\dagger\hat\phi(y)|0} \propto \int \mathcal{D}\phi\, \phi(x)^*\phi(y)e^{iS}.
\end{e}
The absence of hats in the right hand side is not an accident; the $\phi(x)$ in the integral is a number (not an operator) representing essentially the magnitude of the operator $\hat \phi(x)$. That way we can integrate over $\phi$ to produce a number for the expectation value of $\hat \phi(x)^\dagger \hat \phi(y)$ as expected, rather than an operator-valued equation.

Again, the proportionality is present because we have not yet normalized the wavefunction, but we can easily do this by dividing by $\braket{0|0}$. If we also generalize the result to arbitrarily many fields in the left hand side,
\begin{es}
  &\braket{0|\hat\phi(x_1)^\dagger\dots\hat\phi(x_n)^\dagger\hat\phi(y_1)\dots\hat\phi(y_n)|0} \\&=\frac{\int \mathcal{D}\phi\, \phi(x_1)^*\dots\phi(x_n)^*\phi(y_1)\dots\phi(y_n)e^{iS}}{\int \mathcal{D}\phi\, e^{iS}}
  \label{eqn:qft-least-action}
\end{es}

This law, due to Feynman, is the heart of QFT. It can be used to compute the outcome of scattering experiments and particle properties such as magnetic moments Along with the quantitative predictions, (\ref{eqn:qft-least-action}) gives deep insight as to the nature of particles, since it can be used to derive the spin-statistics theorem which explains why spin-1/2 particles obey the Pauli exclusion principle.

\subsubsection*{Low Dimensionality Terms Dominate}
Before we begin investigating this principle of least action, there are two crucial corollaries of the physics we have so far discussed. One is that the effect of complicated terms in a Lagrangian is suppressed, which can be seen with unit analysis. Since the action is unitless in a natural unit system, the Lagrangian must have units of 4. Consider a scattering experiment, where two particles $\phi$ with dimension 1 were collided with energy $E$. If $\mathcal{L}$ had a term like $\lambda \phi^6$, the constant $\lambda$ would need to have dimension $-2$ to satisfy unit analysis. Whenever this constant appeared in the scattering cross section, it would need to be accompanied with an energy to allow the units to work: $\lambda / E^{-2} = \lambda E^2$. This would be measurable at high energies but suppressed at low energies. ``High'' energies here should be compared to the mass energy of a particle, so that $E$ is high only for highly relativistic scenarios. Thus, these complicated terms do not influence our low-energy environment. Throughout this book, we will therefore limit ourselves to as simple Lagrangians as we can find.

\subsubsection*{Charge Parity Time Symmetry}
Another crucial result of the QFT principle of least action is the CPT theorem, which states that if the charge of every particle, the parity of the coordinate system, and the direction of time are all reversed, then the predictions of the QFT principle of least action are unchanged, no matter what the Lagrangian is. This theorem was not understood to be important until well into the development of QFT, but it aids in understanding many of QFT's quirks, including the presence of antiparticles. A proof is not given here, but can be understood from other sources after first gaining some familiarity with particles and spin.

The next part of this book is dedicated to understanding the $n$-point correlation functions, which are the left hand side of (\ref{eqn:qft-least-action}), while the second part focuses on evaluating the right hand side. The third and fourth parts focus mostly on which Lagrangian Nature plugs into (\ref{eqn:qft-least-action}).