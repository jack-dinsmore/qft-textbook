\chapter{Spin One Half}
\label{chap:spin-one-half}

\noindent This chapter deals with extending QFT to cover particles with spin, also known as \emphi{fermions}. Spin is covered in non-relativistic quantum mechanics, though its properties seem sometimes mysterious and arbitrary. In particular, the fact that fermions obey the Pauli exclusion principle and bosons do not cannot be explained in non-relativistic quantum mechanics. This explanation is provided by relativity in QFT in the form of the spin-statistics theorem, which we explain in this chapter. Before arriving at this key result, however, we discuss spin in a generalized sense (section \ref{sec:non-rel-spin}) and show that when relativity is taken into account, spin implies the existence of antiparticles (section \ref{sec:clifford}).

Let's start with a simple definition: spin is the intrinsic angular momentum of a particle. In order to store a particle's spin state in QFT we'll need to graduate our scalar field particle operator $\hat \phi$ to a vector of operators $\hat \psi = (\hat \psi_0, \hat \psi_1,\dots)^T$. This vector is called a \emphi{spinor}.

The next section uses the definition of spin as intrinsic angular momentum to dictate how many entries the spinor must have and what operators can act on it. Our discussion will start with the non-relativistic case, where we do not consider boosts, and then we will add boosts to see how they change the discussion. Following this, we'll describe how to similarly graduate the $\psi$ in the path integral of the QFT principle of least action to a spinor. With this complete, we will repeat the analysis of the previous chapter and derive Feynman rules for fermions. Finally, we will use these new fermions to understand electrons, the Coulomb potential, and atomic nuclei.

\section{Spinors and Spin Operators}
\label{sec:non-rel-spin}

Spin is a quantum state, and in any quantum theory, a quantum state is acted on by operators. It will require the description of new mathematical tools which come from the field of group theory, but by the end we will have a complete description of any particle with spin.

To measure the spin of a spinor $\psi$, we act on $\psi$ with a vector of operators $\bm S$. In order to be interpreted as an angular momentum vector, $\bm S$ must satisfy the defining feature of angular momentum: 

\begin{center}
  \textit{Angular momentum generates rotations.}
\end{center}

The notion of a \emphi{generator} is our first group theoretic term and we will define it in the following paragraph. But first, this definition of angular momentum is different from the classical definition of $\bm L = \bm r \times \bm p$. It is nevertheless a correct definition due to Noether's theorem, which states that an operator's generator is conserved if that operator is a symmetry of the Hamiltonian (appendix \ref{app:noether}). Rotations are a symmetry of spacetime, meaning that if angular momentum generates rotations then angular momentum is conserved. The cross product definition of the angular momentum will return later in this section.

\subsection{Generators}

Now to define what a generator is. Consider a rotation operation $R$ which acts on a three-component vector. We know that $R$ can be written as products of the three-dimensional rotation matrices:
\begin{ec}
  R_x(\theta) = \parens{\begin{matrix}1 & 0 & 0 \\ 0 & \cos\theta & -\sin\theta \\ 0 & \sin\theta & \cos \theta\end{matrix}}\qquad 
  R_y(\theta) = \parens{\begin{matrix}\cos\theta&0&\sin\theta \\ 0&1&0 \\ -\sin\theta & 0 & \cos \theta\end{matrix}}\\
  R_z(\theta) = \parens{\begin{matrix}\cos\theta & -\sin\theta & 0 \\ \sin\theta & \cos \theta & 0 \\ 0&0&1\end{matrix}}.
  \label{eqn:rotation-matrices}
\end{ec}
This fact shows that there are many possible rotation matrices, but that they can each be traced down to the three matrices above. In fact, the above rotation matrices can be simplified even more by writing $R_j(\theta) = e^{i\theta S_j}$, where 
\begin{ec}
  S_x = \frac{i}{2}\parens{\begin{matrix}0 & 0 & 0 \\ 0 & 0 & -1 \\ 0 & 1 & 0\end{matrix}}\qquad 
  S_y = \frac{i}{2}\parens{\begin{matrix}0&0&1 \\ 0&0&0 \\ -1 & 0 & 0\end{matrix}}\\
  S_z = \frac{i}{2}\parens{\begin{matrix}0 & -1 & 0 \\ 1 & 0 & 0 \\ 0&0&0\end{matrix}}.
  \label{eqn:rotation-generators}
\end{ec}
You can check for yourself that exponentiating these matrices gives back the rotation matrices. An arbitrary rotation can therefore be written as 
\begin{e}
  R = e^{i\bm \theta \cdot \bm S}
  \label{eqn:rotation-generation}
\end{e}
where we have arranged $S_x$, $S_y$, and $S_z$ into a vector called $\bm S$. The three components of the vector $\bm \theta$ are called Euler angles and they parameterize the rotation matrix $R$.

In mathematical language, we say that $\bm S$ \emph{generates} the rotations because all rotations can be achieved by plugging different values of $\bm \theta$ into (\ref{eqn:rotation-generation}). The fact that angular momentum is generated by rotation could mean that we should just take $\bm S$ as the spin operator $\bm S$, and write the spinor as a three dimensional vector which $\bm S$ acts on.

\subsection{Angular Momentum Algebra}

This procedure does not work in practice because there are other ways to generate rotations. One can rotate a complex vector for instance, in which case we need a new set of $\bm S$ matrices which are themselves complex. There is an even finer point which must be made: it is necessary that $\bm S$ generate a three-dimensional rotation since spacetime is invariant under three-dimensional rotations. However, the matrix $R$ need not act on a three-dimensional vector. $R$ could act on a five-dimensional vector, for example, and rotate only in a three-dimensional subspace. In this case $\bm S$ would still contain three generators, but each generator would be a $5\times 5$ matrix rather than $3\times 3$.

We therefore need to write a more general definition of rotations than just the rotation matrices (\ref{eqn:rotation-matrices}). One property rotations must have is that they must preserve the length of vectors. That is, $R$ must be unitary. Problem \jtd{cite} asks the reader to show that for $\bm e^{iM}$ to be unitary, $M$ must be hermitian and determinant 1. Thus, the entries of $\bm S$ are hermitian.

The other required property of rotations comes from the way in which they anticommute. Consider rotating a vector aligned with the $z$-axis. If we rotate in the $x$ direction by a small angle $\delta$ first and then $y$ by the same $delta$, we get a different vector than rotating in the $y$ direction first then $y$. This is because the rotation matrices anticommute. Using (\ref{eqn:rotation-matrices}) or drawing pictures that the two rotations differ by a rotation about the $z$ axis by angle $\delta^2$. We can turn this fact into a constraint on $\bm S$ by using (\ref{eqn:rotation-generation}) to define the $x$, $y$, and $z$ rotations. They show that $S_xS_y - S_yS_x = i S_z$.

In principle, we could have aligned our initial vector along any axis $z$, so that the more general version of this constraint is
\begin{e}
  \brackets{S_i, S_j} = i\epsilon_{ijk}S_k
  \label{eqn:angular-momentum-algebra}
\end{e}
where $\epsilon_{ijk}$ is the Levi-Civita symbol. Any set of hermitian, determinant one matrices $\bm S = (S_x, S_y, S_z)$ which satisfies this equation is a valid angular momentum operator. We usually refer to (\ref{eqn:angular-momentum-algebra}) as an \emphi{angular momentum algebra}, and matrices that satisfy it are \emphi{representations} of the angular momentum algebra

\subsection{Using Lie groups to find representations of the angular momentum algebras}
The task of finding matrices $\bm S$ which satisfy (\ref{eqn:angular-momentum-algebra}) (or in mathematical language, finding representations of the angular momentum algebra) has fortunately been solved using group theory. A Lie group is a closed, continuous set of operators, where closed signifies that two operators in the group $U$ and $V$ multiply to produce another operator $W$ which is also in the group. We have unknowingly been working with the group of rotations in three dimensions, which is called the ``special orthogonal group of order three,'' or SO(3). ``Special'' means that the group operators do not change the length of vectors, ``orthogonal'' means that the group operators are real and unitary, and the three indicates that the rotation occurs in three dimensions.

A Lie group satisfies the general result that any group element $G$ can be written as 
\begin{e}
  G = e^{i\bm \theta \cdot \bm S}
  \label{eqn:lie-algebra-generator}
\end{e}
where $\bm S$ are a special set of matrices known as the generators of the Lie group. We used this property in (\ref{eqn:rotation-generation}), but we did not know that it was valid for any Lie groups. The matrices $\bm S$ must satisfy special relations to each other which are dependent on the Lie group the generate; these relations are called a Lie algebra. An example is the angular momentum algebra we discussed above, which we now know as the Lie algebra of SO(3). Mathematicians sometimes label this algebra as $\mathfrak{so}(3)$ but physicists usually use SO(3) to refer to the algebra in addition to the group. We will therefore use the $\mathfrak{so}(3)$ notation in this section only.

Once we specify the dimensions of the spinor which the group elements act on --- say the spinor has dimension $n$, then we can write group elements as $n\times n$ matrices. This is called a representation. Our three-dimensional rotation matrices (\ref{eqn:rotation-matrices}) were a representation of SO(3), and our three-dimensional angular momentum operators (\ref{eqn:rotation-generators}) were representations of its Lie algebra. Matrix exponentiating the generators as in (\ref{eqn:lie-algebra-generator}) can be used to produce the representation of the group if a representation of the Lie Algebra is known.

\subsection{Physical consequences of the group theory explanation for spin}

This new formalism connects directly to the definition used at the beginning of this section --- ``Angular momentum generates rotations.'' The connection is that ``rotations'' refers to the Lie group SO(3), which has the Lie algebra given in (\ref{eqn:angular-momentum-algebra}). The angular momentum operators we're looking for are the rotation generators, which are representations of this Lie algebra.

One reason why this group theory framework is necessary is because (\ref{eqn:angular-momentum-algebra}) is not just the Lie algebra of SO(3). It is also the Lie algebra of SU(2), which is the set of unitary matrices which rotate a two-dimensional complex vector, rather than a three-dimensional real vector. Valid angular momentum operators could therefore be generators of SO(3) or SU(2).

Another use of group theory is the mathematical result that all the faithful\footnote{A faithful representation is one where every group element corresponds to exactly one matrix. For odd dimensions, SU(2) is not faithful because every group element corresponds to two matrices which can be generated using $e^{i\bm \theta \cdot \bm S}$.} representations of SO(3) have odd dimensions, and all the representations of SU(2) have odd\footnote{Except, that is, for the trivial representation, where all the generators are the $1\times 1$ matrix with zero as the only element.} dimensions. Therefore, the spinor which the spin operators act on can have any number of dimensions $n$. If $n$ is even, then one uses representations of $\mathfrak{su}(2)$ as spin operators. If $n$ is odd, one uses representations of $\mathfrak{so}(3)$.

This fact that the parity of $n$ changes what group the spin operators come from has several fascinating properties. Firstly, it means that the SU(2) spins may have different properties than the SO(3) spins. This is manifested in the spin-statistics theorem, which is described in the next section. It also means that since SU(2) acts on complex vectors and SO(3) acts on real vectors. This is crucial because, as noted in the previous chapter, complex particles have an electric charge. Thus, all SU(2) particles are automatically charged. SO(3) particles can be neutral or charged, because one can multiply them by a complex phase.

Another result of group theory is that $\bm S^2 = \frac{n^2-1}{4}\mathds{1}$, where $\bm S^2 = S_x^2 + S_y^2 + S_z^2$. This is interesting because we define the spin of a particle $s$ such that the eigenvalue of the angular momentum operator $\bm S^2$ is $s(s+1)$ so that $s=\frac{n-1}{2}$. Thus, particles with half-integer spin $s$ have spinors of even dimension and are therefore represented by SU(2) and we refer to them as fermions. Integer spin particles, called bosons, have odd dimension spinors and are represented by SO(3).

One final result of group theory is a list of what the spin operators actually are. For zero spin ($s=0$), the spinor has dimension 1, and the operators are all the $1\times 1$ matrices $(0)$. This is an incredibly boring case from the perspective of spin, and it is the reason why we were able to complete all of the previous chapter on spin-zero particles.

Spin one-half particles have $n=2$, so that the spin operators are $2\times 2$ representations of $\mathfrak{su}(2)$ which are the Pauli matrices
\begin{ec}
  \sigma_1 = \parens{\begin{matrix}0&1\\1&0\end{matrix}},\qquad
  \sigma_2 = \parens{\begin{matrix}0&-i\\i&0\end{matrix}},\qquad
  \sigma_3 = \parens{\begin{matrix}1&0\\0&-1\end{matrix}}.
\end{ec}
Non-relativistic quantum mechanics courses usually skip the derivation of spin operators and merely state that they are equal to these Pauli matrices, partially because most of the particles dealt with in non-relativistic quantum mechanics are spin one-half so that the operators for other particles are unnecessary.

Spin one particles have $n=3$, so that the spin operators are $3\times 3$ representations of $\mathfrak{so}(3)$. We have already written these matrices in (\ref{eqn:rotation-generators}) because $n=3$ is the representation used to rotate normal three-dimensional geometric vectors. This means that for spin-one particles (and only spin-one particles), spin is a true geometric vector which rotates normally and need not be understood in a quantum mechanical way. Photons are spin-one particles whose spin corresponds to their polarization, and classical electromagnetism only works to describe photons because of this fact --- their spin behaves like a classical vector and can be treated as such. For this reason, we sometimes call spin-one particles ``vector bosons.'' Vector bosons are able to play a special role in Lagrangians because their spin can be used as a Lorentz index to turn vector-valued quantities into scalar-valued quantities that can go into the Lagrangian. This allows them to act as ``gauge bosons,'' as described in the next chapter.

Throughout the rest of this chapter, we will explore only the spin one-half case, saving spin-one for the next chapter.

\subsection{The Lagrangian of a Massless Fermion}

Since we are specializing to the spin one-half case, the spin operators are the Pauli matrices, which can be combined into a three-dimensional vector $\bm \sigma = (\sigma_1, \sigma_2, \sigma_3)$. To make a relativistic version of the spin operator, we need to graduate $\bm \sigma$ to a four-vector $\sigma^\mu$ by adding a time component, which corresponds to Lorentz boost operator for spin.

This section has so far considered only rotations, so we do not know how operators transform under spin. As a simplification, we'll consider a massless fermion, which can only move at the speed of light. Thus, boosting cannot change its state, and $\sigma^0$ should just be the identity matrix:
\begin{e}
  \sigma^\mu = (\mathds{1}, \bm \sigma).
\end{e}

Now that we have relativistic spin operators, we can derive the Lagrangian of this massless Fermion by writing down all terms that satisfy the symmetries of QFT. Firstly, the Lagrangian must be a real scalar rather than spinor-valued, so that $\psi$ cannot exist alone. It must be paired with another spinor as in $\psi^\dagger \psi$, which indicates the dot product between two spinors\footnote{Traditionally, we use matrix notation for spinors and index notation for spacetime indices. Thus, a spinor dot product is $\psi^\dagger \psi$ but a four-vector dot product is $k^\mu k_\mu$.}.

Another symmetry of the Lagrangian is Lorentz transformations, which implies that all Lorentz indices must be contracted. In the scalar field theory, the only object with Lorentz indices we had access to was the derivative $\del_\mu$, but now we also have the spin operators $\sigma^\mu$. Thus, the available terms are of the form
$$\psi^\dagger \psi \qquad \psi^\dagger \sigma^\mu \del_\mu \psi \qquad \psi^\dagger \del^2 \psi.$$
We did not include $\psi^\dagger \sigma^2 \psi$ because $\sigma^2 = 4\mathds{1}$, so this term is redundant with the first.

The Lagrangian must also be invariant under ``Internal'' Lorentz transformations, which transform spinors rather than spacetime. These transformations are of the form $\psi \rightarrow \Lambda \psi$ where $\Lambda = \expp{\frac{1}{2}\sigma^\mu \omega_\mu}$ where $\omega_\mu$ are arbitrarily chosen constants. Since $\sigma^\mu$ are all hermitian, $\Lambda$ is unitary. Thus, $\psi^\dagger \psi \rightarrow \psi^\dagger \Lambda^\dagger \Lambda \psi = \psi^\dagger \psi$ and all of the above terms are invariant under these internal transformations.

A final symmetry to consider is CPT symmetry, which actually forbids the first term --- specifically, due to parity (P). Spinors are invariant under time (T) symmetry, and $\psi \rightarrow \psi^\dagger$ under charge (C) symmetry, so that $\psi^\dagger \psi$ is invariant under CT. But under parity, angular momentum changes direction because it is a cross product, so the spinor cannot be invariant under P.

The last two terms are allowed by all the symmetries of the Lagrangian, but we usually do not include the last term because its dimension is too high. To compute its dimension, consider the middle term which must have dimension 4 so that $S = \int \mathcal{L} d^4x$ is dimensionless (recall that $d^4 x$ has dimension $-4$). Since $\sigma^\mu$ is dimensionless and $\del_\mu$ has dimension 1, this implies that the spinor $\psi$ has dimension $3/2$, in contrast to a scalar field's dimension $[\phi] = 1$. Therefore, the last term $\psi^\dagger \del^2 \psi$ has dimension 5. To bring it to the correct dimension, we must multiply by a number $g$ which has dimension $-1$, and therefore all the scattering cross-sections that use this term will contain a factor of $(gE)^n$, where $n>0$ is an integer and $E$ is the energy of the scattering. But scattering is generally done at low energies, so that $gE$ is small. Therefore, this term has a small impact on the physics of fermions and we do not include it.

The final Lagrangian for a massless fermion is thus
\begin{e}
  \mathcal{L} = \psi^\dagger \sigma^\mu \del_\mu \psi.
\end{e}
This gives the famous Dirac equation (\ref{eqn:dirac}), when variational calculus is used to compute the corresponding differential equation.


\section{Massive Fermions and the $\gamma$ Matrices}
\label{sec:clifford}

The Lagrangian we just derived does not contain a mass term because we had assumed a massless fermion in the above discussion. We could try to add a mass term of the form $\psi^\dagger \psi$ --- inspired by the $\phi^*\phi$ mass term of a scalar field --- but this is incorrect because $\psi^\dagger \psi$ is not CPT invariant as we discussed above.

The solution is to introduce another particle. We call the original spinor $\psi_L$and we define its transformation under parity as $P\psi_L = \psi_R$. Likewise, $P \psi_R = \psi_L$ since $P^2 = \mathds{1}$. We say that $\psi_L$ has ``left helicity'' and $\psi_R$ has ``right helicity.'' A more practical definition of \emphi{helicity} is that it is the projection of spin against the velocity of the particle\footnote{This definition comes from the fact that we know helicity must flip under parity transformations, and that spin also flips under parity transformations while normal four-vectors do not. So the projection of spin against a four-vector --- of which velocity is the only available one --- flips under helicity}.

The $\psi^\dagger \psi$ (now $\psi_L^\dagger \psi_L$) can be made parity invariant by converting it to $\psi_R^\dagger \psi_L + \psi_L^\dagger \psi_R$. The new Lagrangian is 
\begin{e}
  \mathcal{L} = \psi_L^\dagger \sigma^\mu \del_\mu \psi_L + \psi_R^\dagger \overline \sigma^\mu \del_\mu \psi_R + m(\psi_R^\dagger \psi_L + \psi_L^\dagger \psi_R)
\end{e}
where $\overline \sigma^{\mu} = (\mathds{1}, -\bm \sigma)$ is the parity reflection of $\sigma^\mu$ and $m$ is the mass.

It is convenient to combine $\psi_L$ and $\psi_R$ into one spinor called the Dirac spinor and the spin matrices into four-by four matrices called the Dirac $\gamma$ matrices:
\begin{e}
  \hat \psi = \parens{\begin{matrix}\hat\psi_L \\ \hat\psi_R^\dagger \end{matrix}}.
\end{e}
\begin{e}
  \gamma^\mu = \parens{\begin{matrix}0 & \sigma^\mu \\ \overline \sigma^\mu & 0\end{matrix}}.
  \label{eqn:weyl-rep}
\end{e}
The $\psi_L$ and $\psi_R$ components can be extracted back out of $\psi$ via another matrix
\begin{e}
  \gamma^5 = \parens{\begin{matrix}\mathds{1} & 0 \\ 0 & -\mathds{1} \end{matrix}}.
  \label{eqn:gamma-5}
\end{e}
because the eigenvectors of this matrix are $\hat \psi_L$ and $\hat \psi_R$, with eigenvalues $+1$ and $-1$ respectively. So this matrix separates the left and right components. In particular, $\frac{1 + \gamma_5}{2}$ selects out the left-handed component and removes the right-handed component, which will be crucial when we discuss the weak force.

With these changes in notation, the Lagrangian is now written as
\begin{e}
  \mathcal{L} = \psi^\dagger \gamma^0 (\gamma^\mu \del_\mu - m) \psi.
\end{e}
In an effort to simplify things even further, a common approach is to define the new symbols $\overline \psi = \psi^\dagger \gamma^0$ and $\gamma^\mu \del_\mu = \slashed \del$. Then
\begin{e}
  \mathcal{L} = \overline \psi (i\slashed \del - m) \psi.
  \label{eqn:free-fermion-lagrangian}
\end{e}
This is called the \emph{Dirac Lagrangian} and it is a fully consistent relativistic theory of a fermion.
\jtd{Some $i$ factors missing.}

The remainder of this book can be understood with just this information, but the reader might be curious as to whether introducing a new particle $\psi_R$ was the only resolution to the problem of CPT invariance. A fascinating way to answer this question is to derive the Dirac Lagrangian and the $\gamma$ matrices from a different perspective: we will find the generators of the Lorentz transformations. Lorentz transformations take the form
\begin{e}
  {\Lambda^\mu}_\nu = \parens{
    \begin{tabular}{c|ccc}
      Time dilation & & Boosts & \\\hline
       & & & \\
      Boosts & & Rotations & \\
       & & & \\
    \end{tabular}
  }
  \label{eqn:boost-picture}
\end{e}
so that these generators will include the generators of rotation, which are the Pauli spin matrices. However, the additional generators due to boosts will force us to use the $\gamma^\mu$ matrices instead of the $\sigma^\mu$ matrices. The use of four-component spinors is therefore required for massive fermions.

Along the way, we will discover many useful properties of the $\gamma^\mu$ matrices, find how spinors transform under Lorentz transformations, and show that many forms of the $\gamma^\mu$ matrices exist beyond (\ref{eqn:eqn:weyl-rep}).

\subsection{Generators of the Lorentz Group}

A full four-dimensional sketch of boosts is given in (\ref{eqn:boost-picture}), and an example of a boost in two dimensions is 
\begin{e}
  \Lambda = \parens{\begin{matrix}
    \cosh \gamma & \sinh\gamma\\
    \sinh \gamma & \cosh\gamma\\
  \end{matrix}}
\end{e}
Our first task is to add generators for boosts to our rotation generators in order to create the full set of generators of the Lorentz boosts $\Lambda$. A glance at the above example shows that the boosts are almost rotations, but they are symmetric and the trigonometric functions are replaced with hyperbolic trigonometric functions. A little more thought reveals that the additional generators for the boosts are 
\begin{ec}
  B_x = i\parens{\begin{matrix}0&1&0&0\\-1&0&0&0\\0&0&0&0\\0&0&0&0\\\end{matrix}},\qquad
  B_y = i\parens{\begin{matrix}0&0&1&0\\0&0&0&0\\-1&0&0&0\\0&0&0&0\\\end{matrix}},\\
  B_z = i\parens{\begin{matrix}0&0&0&1\\0&0&0&0\\0&0&0&0\\-1&0&0&0\\\end{matrix}},
  \label{eqn:boost-generators}
\end{ec}
which can be used to verify the example. To write an algebra that contains both the boost generators $\bm B$ and the rotation generators $\bm S$ we define an antisymmetric tensor of matrices $\Sigma$ where the entries are generators. The timelike entries are boosts $\Sigma^{0i} = B^i$ and the spacelike components are rotations $\Sigma^{12} = S_z$, $\Sigma_{13}=-S_y$, and $\Sigma_{23}=S_x$\footnote{This structure of the $\Sigma$ tensor may seem odd, but it is very similar to the layout of the electromagnetic tensor, also known as the Faraday tensor, where the electric field is laid along the timelike components and the magnetic field along the spacelike components.}. The $\Sigma$ tensor then satisfies the algebra
\begin{e}
  \brackets{\Sigma^{\mu\nu},\Sigma^{\rho\sigma}} = i\parens{\eta^{\nu\sigma}\Sigma^{\mu\rho} + \eta^{\mu\rho}\Sigma^{\nu\sigma} -\eta^{\mu\sigma}\Sigma^{\nu\rho} - \eta^{\nu\rho}\Sigma^{\mu\sigma}}
  \label{eqn:lorentz-algebra}
\end{e}
where $\eta^{\mu\nu}$ is the spacetime metric. This equation is called the \emphi{Lorentz algebra}.

We have done nothing except put the spin operators in relativistic language by adding in boost generators so far. Because there are now six generators, the $\Sigma$ tensor can no longer be considered angular momentum. However, we can use a trick due to Dirac to reduce the tensor to four components by defining a four-vector of matrices $\gamma^\mu$ such that
\begin{e}
  \Sigma^{\mu\nu} = -\frac{i}{4}\brackets{\gamma^\mu, \gamma^\nu}.
  \label{eqn:def-gamma-matrices}
\end{e}
Then the Lie algebra of $\Sigma^{\mu\nu}$ implies the following Lie algebra for $\gamma^\mu$:
\begin{e}
  \braces{\gamma^\mu, \gamma^\nu} = 2\eta^{\mu\nu}
  \label{eqn:clifford-algebra}
\end{e}
where $\braces{\cdot,\cdot}$ is the anticommutator. (\ref{eqn:clifford-algebra}) is called the \emphi{Clifford algebra} or the Dirac algebra, and its representation $\gamma^\mu$ are called the Dirac $\gamma^\mu$ matrices. One can manually check that (\ref{eqn:weyl-rep}) is a valid representation of the $\gamma^\mu$ matrices --- i.e., it satisfies th Clifford algebra. Any other representation that satisfies the Clifford algebra is also valid but we will not need them for this book. Likewise, there is no point in giving the additional $6\times 6$ and higher-dimensional representations of the Clifford algebra because we will not use them, but they do exist.


\subsection{Lorentz Boosting Spinors}
We have defined the $\gamma^\mu$ matrices with a spacetime index, so far just because there were four $\gamma^\mu$ matrices and spacetime is four-dimensional. However, this index is not just for show. It is a true spacetime index in that it indicates that the $\gamma^\mu$ matrices transform as vectors under Lorentz transformations.

To see this, we will first act on the $\gamma^{\mu}$ matrices with the spin representation of a boost $\Lambda$. This is sometimes called an ``internal boost'' because it affects only the spin. We can write $\Lambda$ as an exponential using the $\Sigma^{\mu \nu}$ generators of the Lorentz algebra which we just worked out:
\begin{e}
  \Lambda = e^{-i \theta{\mu\nu}\Sigma^{\mu\nu}}.
\end{e}
In principle, $\theta_{\mu\nu}$ can be any tensor of constants. However, because $\Sigma^{\mu\nu}$ is antisymmetric, only the antisymmetric part of $\theta_{\mu\nu}$ will matter. Acting with $\Lambda$ on the $\gamma^\mu$ matrices gives the transformation 
\begin{e}
  \gamma^{\mu} \rightarrow \Lambda^{-1} \gamma^{\mu} \Lambda.
\end{e}
Remember that the $\Sigma^{\mu\nu}$ matrices do not commute with $\gamma^{\mu}$.

If we were to transform the $\gamma^\mu$ as a spacetime vector rather than as a set of matrices, then we would get
\begin{e}
  \gamma^{\mu} \rightarrow {\Lambda^\mu}_\nu \gamma^\nu.
  \label{eqn:gamma-transformation}
\end{e}
where this time, ${\Lambda^\mu}_\nu$ is a boost of spacetime, meaning that its entries are numbers and not matrices. This is called an ``external boost'' because it explicitly does not affect spin.

It helps to write the ${\Lambda^\mu}_\nu$ matrix as an infinitesimal boost, in which case we can use the generators of the Lorentz algebra that we derived above in (\ref{eqn:boost-generators}), which we now call $J^{\mu\nu}$ to distinguish them from the generalized spin operators $\Sigma^{\mu\nu}$. Then ${\Lambda^\mu}_\nu = \delta^\mu_\nu + {\parens{\frac{1}{2}\omega_{\alpha\beta} J^{\alpha\beta}}^\mu}_\nu$. A helpful identity of $\gamma^\mu$ matrices is that
\begin{e}
  \brackets{\gamma^\alpha, \Sigma^{\mu \nu}} = {(J^{\mu \nu})^\alpha}_\beta \gamma^\beta
  \label{eqn:gamma-sigma-commutator}
\end{e}
Using this fact, a few calculations using the Lorentz algebra (\ref{eqn:lorentz-algebra}) and Clifford algebra (\ref{eqn:clifford-algebra}) give that the internal and external boosts are identical: that is,
\begin{e}
  \Lambda^{-1}\gamma^{\mu} \Lambda = \Lambda{^\mu}_\nu \gamma^\nu.
\end{e}
This confirms that $\gamma^{\mu}$ transforms like a four-vector.

\subsection{Finding a Lorentz Invariant Mass Term}
We have now derived the $\gamma^\mu$ matrices and shown that they can be contracted with $\del_\mu$, reproducing the massless Dirac Lagrangian. In our previous discussion, we needed to define $\overline \psi = \psi^\dagger \gamma^0$ in order to include a mass term so that $\overline \psi \psi$ is Lorentz invariant. The reason why $\psi^\dagger \psi$ is not Lorentz invariant can now be seen mathematically from the fact that $\gamma^\mu$ are not all hermitian so that $\Lambda$ is not unitary. Thus, $\psi^\dagger \psi \rightarrow \psi^\dagger \Lambda^\dagger \Lambda \psi \neq \psi^\dagger \psi$ under a Lorentz transformation.

Inserting a $\gamma^0$ matrix, the true mass term transforms as $\overline \psi \psi \rightarrow \psi^\dagger \Lambda^\dagger \gamma^0 \Lambda \psi$. We'll show that $\Lambda^\dagger \gamma^0 \Lambda = \mathds{1}$ with (\ref{eqn:gamma-sigma-commutator}). Writing out an infinitesimal Lorentz transformation,
\begin{e}
  \overline \psi \psi \rightarrow \overline \psi \psi + i\theta_{\mu\nu}\psi^\dagger \parens{\Sigma^{\mu\nu}\gamma^0 - \gamma^0(\Sigma^{\mu\nu})^\dagger}\psi
\end{e}
The goal is to show that 
\begin{e}
  \Delta^{\mu\nu} = i\psi^\dagger \parens{\Sigma^{\mu\nu}\gamma^0 - \gamma^0(\Sigma^{\mu\nu})^\dagger}\psi
\end{e}
is zero.

If $\mu$ and $\nu$ are both spacelike, then $(\Sigma^{\mu\nu})^\dagger = \Sigma^{\mu\nu}$ and the term in parentheses is just the commutator given in (\ref{eqn:gamma-sigma-commutator}). $J^{\mu\nu}$ for spacelike indices are the rotation generators and never have timelike components, so that $\Delta^{\mu\nu} = 0$. If one is spacelike and the other is timelike, then $(\Sigma^{\mu\nu})^\dagger = -\Sigma^{\mu\nu}$ for boosts, but (\ref{eqn:gamma-sigma-commutator}) implies that $\Sigma^{\mu\nu}$ and $\gamma^0$ anticommute, so again $\Delta^{\mu\nu} = 0$. Thus, $\overline \psi \psi = 0$.

\section{Spin-Statistics Theorem: Fermions Anticommute}
In most quantum mechanics courses, one of two definitions is given for a fermion. The first is the definition given in the last section --- that fermions are particles with half-integer spin. The second is that fermions obey the Pauli exclusion principle, meaning that no more than one can occupy a single quantum state.

Why these two definitions are equivalent is a mystery until the presence of relativity and CPT symmetry, in which case Pauli exclusion and half-integer spins are tied together in the \emphi{spin-statistics theorem}. In this section, we prove the spin-statistics theorem using a method due to Julian Schwinger. This gives us insight not only into the deep nature of Fermions, but also how to think of the Pauli exclusion principle for a free particle. We will use this insight to clarify the path integral definition in the case of fermions before deriving the fermionic Feynman rules in the next section.

The proof of the spin statistics theorem is as follows. Half-integer spin particles transform via SU(2) under a rotation whereas whole-integer spins transform via SO(3) as discussed in section \ref{sec:non-rel-spin}. A key difference between these two groups is that SU(2) is ``twice as big,'' meaning that a rotation in SO(3) is contained twice in SU(2). This is essentially because a rotation $R$ in SO(3) can be represented with real matrices, whereas the same rotation $R$ in SU(2) is represented with complex matrices. The complex conjugate of the complex matrices is another member of SU(2) but is the same rotation, so that $R$ is contained twice.\jtd{Is this true?}

A consequence of this fact is that if we apply a full SO(3) rotation to spacetime, the spin (transforming under the same rotation via \ref{eqn:rotation-generation}) completes only a half-turn. We would need another full turn to cover the entire SU(2) spin group.

Suppose we have two spinors which are on opposite sides of the origin --- $\hat \psi(x) R\hat \psi^\dagger(-x)$ --- where one is rotated by a half-turn $R$. Another rotation by $R$ switches their location while also rotating the spinors: $R\hat \psi(-x) R^2\hat \psi^\dagger(x)-R\hat \psi(-x) \hat \psi^\dagger(x)$. CPT inverting this expression gives $-R\hat \psi^\dagger(-x) \hat \psi(x)$. Since QFT is CPT and rotation invariant, the first and the last expressions are equal:
\begin{e}
  \hat \psi(x) R\hat \psi^\dagger(-x) = -R\hat \psi^\dagger(-x) \hat \psi(x).
\end{e}
Due to translation invariance, we could have placed these spinors at any pair of points $x$ and $y$. It therefore follows that $\hat \psi(x)\hat\psi(y) + \hat \psi(y)\hat\psi(x) = 0$, or
\begin{e}
  \braces{\hat \psi(x), \hat \psi(y)} = 0.
\end{e}
In other words, fermions \emph{anti-commute}. This is the opposite of bosons, where $R^2\hat \psi = \psi$ and we have
\begin{e}
  \brackets{\hat \phi(x), \hat \phi(y)} = 0
\end{e}
so that bosons \emph{commute}. The anticommuting nature of fermions is directly responsible for the Pauli exclusion principle since, for $x=y$, it implies $\phi(x)\phi(x)^2 = 0$. Therefore, the operator that creates two particles at position $x$ sends the vacuum to zero and these particles cannot be created.

\section{Feynman Rules for Fermions}
Now that we have the Lagrangian of free fermions (\ref{eqn:free-fermion-lagrangian}) , we can solve the path integral to find $n$PCFs. This path integral is
\begin{ec}
  \braket{0|\hat\psi(k_1)\cdots \hat \psi(k_n)\hat\psi^\dagger(q_1)\cdots \hat \psi^\dagger(q_m)|0} = \frac{\int \mathcal{D}\overline \psi \mathcal{D}\psi\, \psi(k_1)\dots \psi^*(q_m)e^{iS}}{\int \mathcal{D}\overline \psi \mathcal{D}\psi\, e^{iS}}\\
  S = \int d^4 x\, \overline \psi (i\slashed \del - m)\psi
\end{ec}
where $\psi$ represents a four-component vector of complex numbers\footnote{$\psi$ has four components, but it is not a four-vector. It is a spinor. Four-vectors and spinors transform differently under boosts and should be regarded as separate, as described in section \ref{sec:clifford}}.
The eagle-eyed may notice that this equation has a deep flaw, which is that the $\hat \psi$ operators anticommute due to the spin-statistics theorem, but the $\psi$ complex numbers in the path integral commute just as normal numbers do, which is inconsistent. This is fixed by defining the $\psi$s in the integrand to be a new type of number --- Grassmann numbers --- which anticommute just like the $\hat \psi$ operators do. The properties of Grassmann numbers are outlined in appendix \ref{app:grassmann}, but the only property we will really need is that Grassmann numbers still follow Wick's theorem.

For scalar fields in chapter \ref{chap:spin-zero}, we used Wick's theorem (\ref{eqn:wicks-theorem}) to write an $n$PCF in terms of the Green's function of the Lagrangian, $G(k)$. Since Wick's theorem holds for fermions too, all we need is this Green's function for the fermion Lagrangian.

Recall that the position-space Green's function $G(z)$ for a differential operator $O_x$ is defined as $O_x G(z) = \delta(z)$. We showed in section \ref{sec:scalar-free-path-integral} that, when $O_x$ is symmetric under translations, the Green's function in momentum space is just given by $O(k)G(k) = 1$, where $O(k)$ is the original differential operator with $\del_\mu$ replaced by $-ik_\mu$. For fermions, the differential operator is $O_x = (i\slashed \del - m)$, so $O(k) = (\slashed k - m)$ and
\begin{e}
  G(k) = \frac{i}{\slashed k - m-i\epsilon} = \frac{i(\slashed k + m)}{(\slashed k)^2 - m^2-i\epsilon} = \frac{i(\slashed k + m)}{k^2 - m^2-i\epsilon}
  \label{eqn:fermion-propagator}
\end{e}
\jtd{Why the i?}
where we used the fact that $(\slashed k)^2 = k_\mu k_\nu \gamma^\mu \gamma^\nu$ and the the Clifford algebra (\ref{eqn:clifford-algebra}) to show that $(\slashed k)^2 = k^2$. The $i\epsilon$ term is inserted to remove the singularity which occurs for $k^2 = m^2$, just as we did for the scalar field case.

Now that we've worked out the fermionic propagator, we can use the same diagram technology as for scalar fields to compute scattering amplitudes for fermions. The only changes come from the fact that the $\hat \psi$ operators anticommute, and from the existence of $\gamma^\mu$ matrices in the propagator. These changes are summarized below and explained in the following paragraphs. Examples are also given in the next section.
\begin{enumerate}
  \item The propagators are directed. We should take the non-reversibility of these propagators into account when computing symmetry factors (since fermions are charged)
  \item The propagators should be multiplied in order when traversing backwards along charge arrows (since $\gamma^\mu$ matrices do not commute)
  \item If two diagrams are equivalent under odd permutations of the external vertices, they contribute opposite signs\footnote{An odd permutation is a permutation which contains an odd number of swaps.} (since fermions anticommute)
  \item To sum over all possible spins, take the trace over products of $\gamma^\mu$ matrices (since the trace is rotation-invariant)
  \item We should trace over all propagators inside a loop and multiply by $-1$ (since fermions anticommute)
\end{enumerate}

Change one is due to the fact that $\psi=\overline \psi$, which is clear from the fact that $\psi$ is a column vector and $\overline \psi$ is a row vector. They are different quantities and cannot be permuted. This is similar to the case of a complex scalar field, where $\psi$ and $\psi^*$ could not be permuted.

Change two is due to the convention that multiplication is conducted from right to left. We wrote the $n$PCF with the initial states on the left and the final states on the right, so we must also multiply the propagators in this order. This matters because $\gamma^\mu$ matrices do not commute, whereas for the scalar theory the propagator was a complex number and multiplication order did not matter.

Change three is due to the fact that fermions anticommute. External vertices in diagrams represent the $\hat \phi(k_i)$ in $n$PCFs, so permuting external vertices is equivalent to permuting spinors. Odd permutations result in an overall minus sign, so odd and even permutations should possess different signs. The overall sign doesn't matter because the phase of a scattering amplitude is not observable.

Change four deserves some more explaining. A cross section gives the probability to scatter from one state to another, and for fermions, those states include spin. Mathematically, one should include spin by multiplying each external propagator (which is a $4\times 4$ matrix) by a four-component vector indicating the spin state of the external particle. It is common to represent the incoming particle as $u$ and the outgoing particle as $\overline u$ (or $v$ and $\overline v$ for antiparticles). Then the cross-section is $uM\overline u$ where $M$ is the product of the propagators between the particles\footnote{This spin state is usually obtained by first writing $u$ in the particle's rest frame and then boosting it via the transformation laws for spinors discussed in section \ref{sec:clifford}. In the rest frame, the left-handed and right-handed spin states are the same two-component vector, analogous to the non-relativistic, two-component particle spin. After the boost, the left- and right-handed components differ.}. However, most experiments cannot detect spin, so that all possible final $u$ vectors should be \emph{summed} over. They also cannot produce particles of one spin, so that all possible $\overline u$ vectors should be \emph{averaged} over; Since spin vectors are ``complete'' in that they fully cover the four-dimensional unit sphere on which spin states live, the sum of $uM\overline u$ over their values is equal to $\tr (M)$. The average is $\tr (M) / 4$ since there are four independent spin states.

The trace in the fifth change follows from the fourth change; a loop in a diagram represents the spontaneous creation and annihilation of a fermion, which may have any spin. We should therefore sum over all spins by taking the trace of the propagators of the loop. The minus sign comes from the fact that an implicit commutation of fermions must occur to calculate the loop. The Green's function which we use as a propagator is a 2PCF where the integrand is $\psi \overline \psi e^{iS}$. The order of $\psi$ and $\overline \psi$ comes from the time ordering symbol of the $n$PCFs; the early time operator must come first. But when calculating a larger diagram, we traverse propagators backwards according to change number one. Thus, the integrand is $(\overline \psi_1 \psi_1) (\overline\psi_2\psi_2)\dots(\overline\psi_n\psi_n) e^{iS}$, where the subscript indicates which number propagator the fermions belong to. Wick's theorem requires us to write this integrand as a product of $\psi \overline \psi$s so that we can represent the integrand as a product of propagators, but this requires us to compute the last $\psi_n$ to the front --- $(\psi_n\overline \psi_1) (\psi_1 \overline\psi_2)(\psi_2\dots\overline\psi_n) e^{iS}$ --- which is an odd number of permutations. The loop therefore contains a negative sign.

This gives us all the mathematical machinery needed to compute scattering amplitudes for a fermionic theory. For the rest of the chapter, we will work with a common interaction mechanism between scalars and fermions known as Yukawa coupling, using this machinery to compute the consequences the theory. The next section introduces Yukawa coupling and computes several simple diagrams showing particle decay to practice with the above changes to the Feynman rules.

\section{Yukawa Theory: Decay in the Massive Scalar Limit}

The Yukawa theory consists of a fermion and a real scalar particle together. The Lagrangian for such a theory is simply the sum of the scalar and fermion Lagrangians. We also add an interaction between the two particles, which goes as $\overline \psi \psi \phi$. This kind of coupling is called \emphi{Yukawa coupling}. The full Lagrangian is
\begin{e}
  \mathcal{L} = \frac{1}{2}|\del_\mu \phi|^2 - m_\phi \phi^2 + \overline \psi (i\slashed \del - m_\psi)\psi + g \overline \psi \psi \phi
  \label{eqn:yukawa}
\end{e}
The Yukawa coupling is actually the only way we could couple a fermion to a scalar. $\phi \psi$ is disallowed because the spinor $\psi$ is not dotted with anything and the Lagrangian cannot be spinor-valued. On the other hand, $\overline \psi \psi \phi$ already has dimension 4 so any additional particles would make it an irrelevant operator. This is one reason why Yukawa interactions are so studied. Another is that they appear in the Standard Model, as will be discussed later.

We'll represent the scalar $\phi$ particle as a dashed line and the fermion $\psi$ as a solid line. The vertex corresponding to this new interaction is
\begin{center}
  \feynmandiagram [horizontal=a to b, scale=0.8] {
    i1 -- [fermion] b -- [fermion] i2,
    a -- [scalar] b,
  };
\end{center}
and contributes a factor of $-ig$ to the amplitude.

In this section, we'll assume a massive scalar: $m_\phi \gg m_\psi$. Intuitively, we know that having a massive scalar implies that the decay of one $\phi$ particle to two (or more) $\psi$ particles is energetically possible. The rest of this section studies this decay.

The leading order diagram for $\phi\rightarrow \overline\psi\psi$ decay is just the interaction vertex
\begin{center}
  \feynmandiagram [horizontal=a to b, scale=0.8] {
    i1 [particle=$\overline \psi$] -- [fermion] b -- [fermion] i2 [particle=$\psi$],
    a [particle=$\phi$] -- [scalar] b,
  };
\end{center}
which has no internal legs, so it merely has a scattering amplitude of $-ig$. The next order diagrams are more interesting:
\begin{center}
  \feynmandiagram [horizontal=a to b] {
    i1 [particle=$\overline \psi$] -- [fermion, momentum=$q_1$] b -- [fermion, momentum'=$p + q_1$] s1 -- [fermion, momentum=$q_2$] i2 [particle=$\psi$],
    a [particle=$\phi$] -- [scalar, momentum=$p$] b,
    s1 -- [scalar, momentum'=$q_4 - q_3$] s2,
    f2 [particle=$\overline \psi$] -- [fermion, momentum=$q_3$] s2 -- [fermion, momentum=$q_4$] f1 [particle=$\psi$],
  };
  \feynmandiagram [horizontal=a to b] {
    i1 [particle=$\overline \psi$] -- [fermion, momentum=$q$] s1 -- [fermion, momentum=$q-k$] b -- [fermion, momentum=$q'-k$] s2 -- [fermion, momentum=$q'$] i2 [particle=$\psi$],
    a [particle=$\phi$] -- [scalar, momentum'=$p$] b,
    s1 -- [scalar, momentum'=$k$] s2,
  };
\end{center}
The first represents $\phi \rightarrow \overline \psi \psi \overline \psi \psi$ while the second is another $\phi\rightarrow \overline \psi \psi$ diagram. Both now have internal propagators, so their matrix elements are more complicated. The first diagram is
\begin{e}
  -iM_{\phi \rightarrow \overline \psi \psi \overline \psi \psi} = \frac{1}{4}(-ig)^3 \frac{i\tr(\slashed p + \slashed q_1 - m_\psi)}{(p+q_1)^2 - m_\psi^2}\frac{1}{(q_4-q_3)^2 - m_\phi^2}
\end{e}
The first fraction is the internal fermion propagator while the second is the internal scalar propagator. The factor of $1/4$ and the trace appear so that $M$ is the spin-averaged cross section. Unfortunately, this diagram evaluates to zero because, as appendix \ref{app:gamma} indicates, the trace of an odd number of $\gamma^\mu$ matrices is zero. Thus, $\phi \rightarrow \overline \psi \psi \overline \psi \psi$ is a higher-order process.

The second diagram is not zero, however. It has three internal legs arranged in a loop, meaning that there is one free momentum to integrate over.
\begin{es}
  -iM_{\phi \rightarrow \overline \psi \psi} = -ig + \frac{(-ig)^3}{4} \int \ftve{4}{k} \tr&\brackets{\frac{i(\slashed q' - \slashed k - m_\psi)}{(q' - k)^2 - m_\psi^2}
  \frac{i(\slashed q - \slashed k - m_\psi)}{(q - k)^2 - m_\psi^2}}\\
  &\times \frac{1}{k^2-m^2_\phi}.
\end{es}
There were two fermion propagators this time, which we computed in backwards order. The trace is another spin average. Using the fact that $\tr(\gamma^\mu \gamma^\nu) = 4\eta^{\mu \nu}$, 
\begin{e}
  -iM_{\phi \rightarrow \overline \psi \psi} = -ig - ig^3 \int \ftve{4}{k} \frac{(q' - k)\cdot(q - k) + m_\psi^2}{((q' - k)^2 - m_\psi^2)((q - k)^2 - m_\psi^2)}\frac{1}{k^2-m^2_\phi}.
\end{e}

\begin{e}
  M_{\phi \rightarrow \overline \psi \psi} = g + g^3 \int \ftve{4}{k} \frac{(q' - k)\cdot(q - k) + m_\psi^2}{((q' - k)^2 - m_\psi^2)((q - k)^2 - m_\psi^2)}\frac{1}{k^2-m^2_\phi}.
\end{e}
Although the integral is complicated, it depends on few variables. Conservation of momentum dictates that $q' + q$ is the initial momentum of the $\phi$ particle, and if we work in the center of mass frame this momentum is $q' + q = (m_\phi, \bm 0)$. Thus, $q = (m_\phi/2, \bm q)$ and $q' = (m_\phi/2, -\bm q)$ where $m_\phi^2 / 4 - (\bm q)^2 = m_\psi^2$. Thus, $M_{\phi \rightarrow \overline \psi \psi}$ depends only on $m_\psi$ and $m_\phi$. 

To leading order in $m_\psi/m_\phi \ll 1$, the cross section is
\begin{e}
  M_{\phi\rightarrow \overline \psi \psi}^\mathrm{2D} \approx g + \frac{g^3}{8\pi}\parens{\frac{m_\psi}{m_\phi}}^2\qquad
  M_{\phi\rightarrow \overline \psi \psi}^\mathrm{3D} \approx g + \frac{g^3}{8\pi}\parens{\frac{m_\psi}{m_\phi}}
\end{e}
in two and three spacetime dimensions respectively. These answers can be confirmed numerically, or computed analytically using the tools of chapter \ref{chap:qed}. Using the conversion from scattering amplitude to decay rate given in (\ref{eqn:decay-rate}), we can triumphantly write our first lifetime for a particle:
\begin{e}
  \tau_\phi^\mathrm{2D} = \frac{2m_\phi}{g}\parens{1 - \frac{g^2}{8\pi}\frac{m_\psi}{m_\phi}}\qquad
  \tau_\phi^\mathrm{3D} = \frac{2m_\phi}{g}\parens{1 - \frac{g^2}{8\pi}\parens{\frac{m_\psi}{m_\phi}}^2}
  \label{eqn:low-dimension-phi-lifetime}
\end{e}
\jtd{This has the wrong dimensions}
Quantum field theory therefore successfully describes particle decay --- the first ever consistent theory of physics to do so.

We have given these answers in unphysically low dimensionality because $M_{\phi \rightarrow \overline \psi \psi}$ diverges as the number of spacetime dimensions approaches four. This divergence also occurred in the case of the scalar particle in chapter \ref{chap:spin-zero}, and in general occurs for a particle which has a dimensionless coupling constant ($\lambda$ for scalar theory; $g$ for Yukawa theory). This was a serious problem historically in developing QFT as mentioned in chapter \ref{chap:intro}, eventually fixed using the techniques we introduce in chapter \ref{chap:qed}. The divergences also have a physical origin which can be understood already as follows:

At the beginning of this section, we justified the Yukawa Lagrangian by stating that any additional interactions are so high in dimension that their cross sections would be proportional to energy (or a positive power of energy). When using two or three spacetime dimensions, even the Yukawa coupling becomes too high in dimensions and we see that $M_{\phi \rightarrow \overline \psi \psi}$ is proportional to $1/m_\phi^{4-d}$, where $d$ is the number of spacetime dimensions. Since $1/m_\phi$ is small, this confirms our intuition that high-dimensionality interactions are irrelevant in this scenario.

But in the four dimensional case, this trend requires that $M_{\phi \rightarrow \overline \psi \psi}$ is proportional to $1/m_\phi^0$ --- that is, it is a pure number depending only on $g$. However, this cannot be true since $m_\phi$ cannot decay if $m_\phi < 2m_\psi$ due to energy conservation\footnote{Even the low-dimensional lifetimes of (\ref{eqn:low-dimension-phi-lifetime}) predict decay for $m_\phi < 2m_\psi$, which is incorrect by this logic. But they were computed to leading order in $m_\phi/m_\psi$, and the higher orders will resolve the conflict by forcing the lifetime to blow up as $m_\phi$ nears $2m_\psi$. This defense mechanism cannot occur in four spacetime dimensions because $m_\phi$ and $m_\psi$ do not appear in the lifetime.}. Its lifetime therefore must depend on $m_\phi$. The divergence of the amplitude is a mathematical indication of this contradiction.

The decay studied in this section occurs in the Standard Model, where the Higgs boson (mass $\SI{125}{\giga \electronvolt}$) takes the role of the scalar particle and the bottom quark (mass $\SI{4}{\giga \electronvolt}$) takes the role of the fermion (though many other decay paths are possible). The decay is so quick that the Higgs has a lifetime of a few times $\SI{e-22}{\second}$, so that it would travel only a few proton widths before decaying if moving at the speed of light, making it difficult to observe.

Since we've already seen these diverging amplitudes in two theories, we will soon describe how to resolve them. Before this, however, we have two more problems to discuss. One is to further explore the properties of fermions in the remainder of this chapter. The other is to describe spin-one particles, such as light, in the next chapter.


\section{Yukawa Potential \& the Meson Model of the Strong Nuclear Force}
In this section, we'll continue using the Yukawa Lagrangian (\ref{eqn:yukawa}) of the previous chapter, but now we'll consider the opposite limit of the mass ratio, where $m_\psi \gg m_\phi$. We will see that in this limit, the bosonic $\phi$ particle acts as a carrier of a long-distance, attractive force between the fermions which constituted the first model of the atomic nucleus. The mechanism of producing long range forces with bosons occurs quite often in the Standard Model; of the five known fundamental bosons, four are force carriers. Electromagnetism is the most famous example of a long-range force carried by the massless photon. The exception is the Higgs boson, which is quite massive and therefore decays before it carries forces, as discussed in the previous section. Light bosons have use even outside QFT; other theories of physics that use field theory techniques sometimes introduce a light boson to carry long-range forces.

The interaction we'll consider is two heavy fermions which interact by passing a boson from one to the other:
\begin{center}
  \feynmandiagram [horizontal=a to b, scale=0.8] {
    i1 -- [fermion,momentum'=$p'$] a -- [fermion,momentum=$p$] i2,
    a -- [scalar,momentum=$p'-p$] b,
    i3 -- [fermion,momentum=$q$] b -- [fermion,momentum'=$q'$] i4,
  };
  \hspace{1em}$+$
  \hspace{1em}
  \feynmandiagram [vertical=a to b, scale=0.8] {
    i2 -- [anti fermion,rmomentum'=$p$] a -- [anti fermion,rmomentum=$q$] i1,
    a -- [scalar,momentum=$q-p$] b,
    i3 -- [anti fermion,rmomentum=$q'$] b -- [anti fermion,rmomentum'=$p'$] i4,
  };
\end{center}
These diagrams' amplitudes are simply
\begin{e}
  -iM = \frac{(-ig)^2}{(p'-p)^2-m_\phi^2} + \frac{(-ig)^2}{(q-p)^2-m_\phi^2}
\end{e}
In section \ref{sec:spin-zero-scattering}, we studied similar diagrams and made use of Mandelstam variables (\ref{eqn:mandelstam}) to simplify the expression. There we had three diagrams, but conservation of charge forbids one diagram (the $u$-channel), so we just have
\begin{e}
  M = -ig^2\parens{\frac{1}{s^2-m_\phi^2} + \frac{1}{t^2-m_\phi^2}}.
  \label{eqn:relativistic-amplitude-yukawa}
\end{e}

The physical consequences of this scattering amplitude are not immediately clear, so let us imagine it in the context of an experiment. Suppose that two beams of distinguishable fermions $A$ and $B$ are fired together and the momenta of the final states are measured. A physicist in the 1920s would have analyzed this situation using non-relativistic quantum mechanics, and determined using the Born approximation that 
\begin{e}
  M = -2\pi V(\bm p - \bm q)\delta(E_{\bm p} - E_{\bm q})
  \label{eqn:non-relativistic-amplitude-yukawa}
\end{e}
where $V$ is the Fourier transform of the potential between the particles. This equation was derived in \ref{sec:scattering}. \jtd{I need to mention that time goes up in diagrams. Derive this equation} Equating this amplitude to the QFT amplitude above \jtd{Why the $i$?} connects the relativistic notion of a three-particle Yukawa interaction to the non-relativistic (and familiar) notion of a potential.

Since the particles in our experiment are assumed to be distinguishable, the $t$-channel which mixes particle types should be neglected. Thus, equating (\ref{eqn:relativistic-amplitude-yukawa}) and (\ref{non-relativistic-amplitude-yukawa}) gives
\begin{e}
  V(\bm k) = -\frac{g^2}{\bm k^2 - m_\phi^2}.
\end{e}
This potential can be converted back to position-space using the tools of complex analysis (discussed in chapter \ref{chap:complex-analysis}) to get a spherically symmetric potential:
\begin{e}
  V(r) = -\frac{g^2 e^{-m_\phi r}}{4\pi r}.
\end{e}
This attractive potential is called the \emphi{Yukawa potential}. It dies of quickly for $r \gg m_\phi$, but for $r \ll m_\phi$ it behaves like the $V(r) \sim 1/r$ Coulomb potential. Indeed, if $m_\phi = 0$, the Yukawa potential becomes the Coulomb potential, though this prediction is difficult to achieve in practice because $\phi$ is unlikely to be massless, as discussed in \ref{sec:scalar-mass}\footnote{The true Coulomb potential comes from a three-point interaction between fermions and a spin-one particle, rather than this Yukawa interaction between fermions and a spin-zero particle.}.

The fact that QFT can produce this long-range potential is at first surprising. A QFT Lagrangian can only include interactions between fields at the same spacetime coordinate, making it difficult for $\psi$ fields at different locations to reliably respond to distant fluctuations. But with the addition of a boson light enough to propagate long distances, we showed that the particles may communicate via the boson. Not only does this interaction reproduce the phenomenon of non-relativistic quantum mechanics; it also forces the interactions to be causal, since the force-carrying boson cannot move faster than the speed of light.

With that said, one could have expected that a correct physical theory should predict long-range forces because electromagnetism is a long-range force. But the Yukawa potential is crucially different from electromagnetism because it does not possess charges. All nucleons attract each other, whereas in electromagnetism particles of the same charge repel. This property is necessary to create a nucleus, which are observed to contain hundreds of particles for millennia without ejecting them. In fact, the Yukawa potential was invented specifically for the purpose of modeling the nucleus by Hideki Yukawa in 1935. He predicted the existence of a hitherto undiscovered scalar boson known as a \emphi{meson} to mediate the nucleon-nucleon force termed the \emphi{strong force}, or strong nuclear force. Soon after his prediction, the muon was discovered and thought to be the meson in question, though it was later determined to be an unrelated particle\footnote{In \textit{The Feynman Lectures}, Richard Feynman refers to the muon as the ``mu meson'' because the fact that the muon does not interact through the strong force.}. Soon afterwards a true meson --- the pion --- was discovered and Yukawa won the Nobel prize in 1949.

Somewhat surprisingly, the discoveries continued and many more mesons were discovered until the list became too large to name. This may be surprising because the Yukawa interaction requires only one meson to exist, but now we understand that the Yukawa interaction is approximate. Protons, neutrons, and mesons are really made up of quarks, and the true Feynman diagram of nucleon-nucleon interactions is as follows 

\jtd{Diagram}

which looks like a Yukawa interaction diagram unless the substructure of the proton, neutron, or meson are revealed. The vast family of mesons exist merely because there are many ways to combine quarks into distinct particles.

The accurate theory of quarks and mesons is called Quantum Chromodynamics, or QCD, and we discuss it in chapter \ref{chap:qcd} after we have developed the mathematics necessary to understand them. That theory was layed out int he 70s \jtd{check}, delayed due to its complexity. We'll therefore focus on the simpler theory of light --- and spin-one particles  in general --- in the next chapter, moving on to the first quantum theory of electromagnetism in the following chapter.



\begin{problem}[Verification of the Clifford algebra]
  Show that the Clifford algebra (\ref{eqn:clifford-algebra}) satisfies the Lorentz algebra (\ref{eqn:lorentz-algebra}) under the definition of the $\gamma^\mu$ functions in (\ref{eqn:def-gamma-matrices}).
\end{problem}

\begin{problem}[Relativistic Stern-Gerlach Experiment]

\end{problem}