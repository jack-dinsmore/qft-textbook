\noindent Quantum Electrodynamics, or QED, is the theory of the electron and the photon, and can also be extended to approximately describe the proton. It is the first example of a successful quantum field theory and remains one the most accurate theory measured.

QED contains two types of particles: fermions and gauge bosons. This part will be dedicated to discussing both types of particle, as well as a simpler particle known as a scalar boson. The difference between these particles is codified in what \emphi{quantum numbers} they hold.

A quantum number is a quantity which is conserved as the particle propagates. From Newtonian mechanics, we are aware of momentum as a conserved property --- and therefore a quantum number --- that particles should have. However, quantum mechanics adds other possibilities such as spin for electrons, polarization for photons, helicity for massive particles, and more. Any particle theory must provide a complete description for the evolution of each of these quantum numbers.