\chapter{Spin One}
\label{chap:spin-one}

In the last chapter, we saw that a massless scalar boson can carry a long-range force between two fermions with a potential of $V(r) \propto 1/r$. This is similar to the electric potential, but a magnetic field is missing. In order to achieve a theory for the magnetic field, we propose a new type of particle named a vector boson which consists of a vector field $A^\mu(x)$, on analogy with the four-vector of electromagnetism. This particle can be called ``spin one'' because it transforms as a vector under Lorentz boosts and rotations, and the only SU(2) or SO(3) representation that rotates like a four vector is the spin-one representation. Our only problem is that the Lagrangian for this field is so high in dimension that $A^\mu$ should not be observable.

We address this dimensionality problem by imposing a gauge symmetry on $A^\mu$ which forbids the offending terms from the Lagrangian. Gauge symmetry is an intrinsic property even of classical electromagnetism, so it is no surprise that it should reappear in QFT. After writing down the possible Lagrangians of $A^\mu$ and how it could couple to a fermion while obeying gauge symmetry, we use the combined vector-fermion theory to reproduce the Coulomb potential between fermions, proving that $A^\mu$ is a photon. Furthermore, the theory predicts that the fermion should have a magnetic dipole moment of about 2, which is true of the electron. Finally, we show that a massless vector particle does not suffer from the same problems as a scalar particle; its masslessness is not ruined by including higher order Feynman diagrams. We're therefore safe to propose that this massless vector particle is the photon, the fermion is the electron, and the theory is a valid theory of Quantum Electrodynamics, or QED.

\section{Gauge Invariance}

For a field $A^\mu$, the simplest kinetic term one can put in the Lagrangian is $\del_\mu A^\mu$, and the simplest potential term is $A^\mu A_\mu$. As with the scalar and fermion fields, we normalize the Lagrangian by giving the kinetic term a coefficient of one while all other terms get their own constant: 
\begin{e}
  \mathcal{L} = \del_\mu A^\mu - m A^{\mu} A_\mu.
\end{e}
where $m$ is the particle mass. The first term requires that $A^\mu$ have dimension 3, so that $m$ has dimension $-2$. Therefore, any dynamics involving this second term are not observable at low energies.

This is a serious problem for any physical model with a vector boson, and seemingly the only solution is to replace the kinetic term $\del_\mu A^\mu$ with a different term, so that the dimensionality of $A^\mu$ could change. But why should $\del_\mu A^\mu$ not exist in the Lagrangian when for the scalar and fermion particles every simple term that could exist in the Lagrangian did so?

We may take inspiration from the classical definition of the vector potential, which is that $A^0$ is the electric potential and $A^i$ or $\bm A$ is the vector potential, meaning the electric and magnetic fields are $\bm E = -\nabla A^0 - \del_t \bm A$ and $\bm B = \nabla \times \bm A$. These $\bm E$ and $\bm B$ fields are observable, while $\bm A$ is not.

Suppose we make the following change to $A^\mu$
\begin{e}
  A^0(x) \rightarrow A^0(x) - \del_t \lambda(x) \qquad \bm A(x) \rightarrow \bm A(x) + \nabla \lambda(x)
  \label{eqn:gauge-symmetry-photon}
\end{e}
where $\lambda(x)$ is an arbitrary function. In spacetime notation, this is $A^\mu(x) \rightarrow A^\mu(x) + \del^\mu \lambda(x)$. Since derivatives commute and the curl of a gradient is zero, the fields $\bm E$ and $\bm B$ are unchanged; that is, there is no physical change to the system. We might say that the above transformation is a ``symmetry'' of the system.

The symmetries we have discussed so far are \index{\emph{global symmetry}}{global symmetries}, meaning that they refer to actions taken on the entire field at once. For example, CPT symmetry is a global symmetry because it states that physics is invariant if one flips the charge, parity, and time of everything in the universe, not just the particles in a single region. (\ref{eqn:gauge-symmetry-photon}) is different because the field $\lambda(x)$ can be different at every position; it may even be zero everywhere except in a small region. We call these local symmetries \emph{\index{gauge symmetry}{gauge symmetries}}.

We've referred to symmetries such as the SU(2) symmetry of spin by their group before. The group for the gauge symmetry of electromagnetism must have only one generator because $\lambda(x)$ was the sole generator of (\ref{eqn:gauge-symmetry-photon}). The only such group is U(1), where the U stands for ``Unitary.'' This group is usually represented as  rotating fields by a phase $e^{i\alpha}$, where $\alpha$ is the Euler angle and the generator is one. In (\ref{eqn:gauge-symmetry-photon}) the representation is more complicated than a phase, but the single generator guarantees that the symmetry group is still U(1).

Enforcing this gauge symmetry of classical electromagnetism as a symmetry of the new quantum field $A^\mu$ will destroy the $\del_\mu A^\mu$ kinetic term. If we perform the transformation of (\ref{eqn:gauge-symmetry-photon}) on $\del_\mu A^\mu$, we get
\begin{e}
  \del_\mu A^\mu \rightarrow \del_\mu A^\mu + \del_\mu\del^\mu\lambda(x)
\end{e}
which is not equal to $\del_\mu A^\mu$. The term is therefore not gauge invariant and cannot be included.

A useful tool for writing gauge invariant equations is to define the \emphi{Faraday tensor}, which is an antisymmetric tensor of fields defined as
\begin{e}
  F^{\mu \nu} = \del^\mu A^\nu - \del^\nu A^\mu
  \label{eqn:faraday-tensor}
\end{e}
which is invariant under Gauge transformations. This allows for a gauge invariant kinetic term $F^{\mu\nu}F_{\mu\nu}$. However, no potential term is possible because $A^\mu A_\mu$ is not gauge invaraint. Thus, our new Lagrangian for a vector boson is
\begin{e}
  \mathcal{L} = -\frac{1}{4}F^{\mu\nu}F_{\mu\nu}
\end{e}
where the $\frac{1}{4}$ factor is a matter of convention. This particle is required to be massless since there is no second term, but in the next section we will show that the addition of an fermion admits another, low-dimensional term, proving that the dynamics of $A^\mu$ are observable.


\section{The Fermion-Vector Boson Lagrangian}
Now that we have understood the importance of respecting Gauge freedom, we can more thoroughly explore the possible Lagrangians of a photon field. The above Lagrangian contains the only two low-dimension terms for a vector particle by itself, but if we include a fermion, things get more complicated. Any term must be invariant under a spin rotation, so it should contain $\gamma^\mu$ and contract $\psi$ with $\overline \psi$. It should also be Lorentz invariant, so all indices should be contracted. The only low-dimension term satisfying both constraints is
$$ie \overline\psi \gamma_\mu A^\mu\psi  = e\overline \psi \slashed{A}\psi $$
where $e$ is a constant with dimension $\frac{1}{2}$. The last thing to check is whether this term is invariant under the new gauge U(1) symmetry, but we cannot do this until we determine how $\psi$ transforms under the gauge symmetry.

In the previous two chapters on fermions and scalars, we required that the Lagrangian be invariant under $\psi \rightarrow e^{i\alpha}\psi$, where $\alpha$ is a constant. This is a global U(1) symmetry, global because $\alpha$ does not depend on position. We also associated this phase with charge, because doing so allowed us to use the CPT theorem to justify that $\psi$ and $\overline \psi$ (or $\phi$ ad $\phi^*$ for complex scalars) are antiparticles. Since we're pursuing a quantum version of electromagnetism, it is very tempting to connect these U(1) global and U(1) gauge symmetries.

Let's promote the U(1) global symmetry of the fermion $\psi$ to a U(1) gauge symmetry. That is, we require that the action be invariant under
\begin{e}
  \psi \rightarrow e^{ie\lambda(x)}\psi
\end{e}
where $\lambda(x)$ is the same $\lambda(x)$ as in (\ref{eqn:gauge-symmetry-photon}). We can then explicitly compute how $e\overline \psi \slashed A \psi$ transforms under gauge transformations:
\begin{es}
  e\gamma_\mu\overline \psi A^\mu \psi \rightarrow\ 
  & e\gamma_\mu \overline \psi\brackets{e^{-ie\lambda(x)}\parens{A^\mu + \del^\mu \lambda(x)}e^{ie\lambda(x)}}\psi\\
  =\ & e\gamma_\mu\overline \psi A^\mu \psi + e\gamma_\mu \overline \psi \psi \del^\mu \lambda(x)\\
\end{es}
which is not gauge invariant. But this promoting the global U(1) to a gauge U(1) also rendered the fermion part of the Lagrangian non-invariant, so we must check how that transforms too:
\begin{es}
  \overline \psi \parens{i\slashed \del - m} \psi \rightarrow\ 
  & i\gamma_\mu \parens{\overline \psi e^{-ie\lambda(x)}} \del^\mu \parens{e^{ie\lambda(x)} \psi} - m \overline \psi e^{-ie\lambda(x)} e^{ie\lambda(x)} \psi\\
  =\ & i\overline \psi \slashed\del \psi + ie\gamma_\mu \overline \psi \parens{i\del^\mu \lambda(x)} \psi - m \overline \psi\psi \\
  =\ & \overline \psi \parens{i\slashed\del - m} \psi - e\gamma_\mu \overline \psi \psi \del^\mu \lambda(x).
\end{es}
This term is also not gauge invariant, but adding it with the previous term causes the extra contribution from the gauge transformations to cancel, so that the sum (and therefore the total action) is gauge invariant.

To reprise, the full Lagrangian for a vector boson plus an electron is
\begin{e}
  \mathcal{L} = -\frac{1}{4}F^{\mu \nu}F_{\mu \nu} + \overline \psi\parens{i \slashed \del - m}\psi + e\overline \psi \slashed A \psi.
\end{e}
though it is sometimes written as
\begin{e}
  \mathcal{L} = -\frac{1}{4}F^{\mu \nu}F_{\mu \nu} + \overline \psi\parens{i \slashed D - m}\psi
\end{e}
where $D_\mu = \del_\mu - i e A_\mu$ is known as the \emph{covariant derivative}.

The same logic also applies to a complex scalar boson, which is why we referred to such particles as charged. There, the Lagrangian is
\begin{e}
  \mathcal{L} = -\frac{1}{4}F^{\mu \nu}F_{\mu \nu} + \frac{1}{2}|D_\mu \phi|^2 - m|\phi|^2
  \label{eqn:qed-lagrangian}
\end{e}
with the same covariant derivative $D_\mu$ as before. One can verify manually that this Lagrangian is also gauge invariant.

For several chapters, we have claimed without justification that complex conjugation represents charge reversal. This is necessary to understand why $\phi$ and $\phi^*$ are antiparticles for a complex scalar field $\phi$, and why $\psi$ and $\overline \psi$ are antiparticles for a fermion $\psi$. U(1) gauge symmetry was the missing link. In the above, we demonstrated that the phase of these particles determines how they couple with the vector boson $A^\mu$, and when we show that $A^\mu$ is a photon, then the coupling constant $e$ will manifestly be the charge of the particle. You can check that if we conjugate $\psi$ by replacing it with $\overline \psi$ in the above argument, the coupling constant $e$ in the covariant derivative becomes $-e$, proving that complex conjugation reverses charge.


\section{Fermions, Gauge Bosons, and Noether's Theorem}
In this chapter we started with a vector boson and introduced a U(1) gauge symmetry, inspired by electromagnetism, in order to make the boson observable. But it is also possible to perform the same task backwards: one can promote any global continuous symmetry to a gauge symmetry and create a vector boson known as a \emphi{gauge boson}. We'll review that process in this section.

When we ``gauged'' the U(1) symmetry of the fermion by promoting its $e^{i\alpha}$ global symmetry of $\psi$ to an $e^{i\lambda(x)}$ gauge symmetry, we made the kinetic term of the fermion no longer gauge invariant. This was because the kinetic term involves a derivative $i\overline \psi \slashed \del \psi$ which acted on the $\lambda(x)$ in the gauge transformation and did not act on the constant $e^{i\alpha}$ global transformation. The same derivative will act on any gauge symmetry, even one which isn't U(1).

In the above section, the gauged symmetry required modifying this offending derivative to a covariant derivative $D_\mu$ which contains a contribution from the new gauge boson. After adding a kinetic term for the gauge boson, the action was gauge invariant.

The form of $D_\mu$ derives directly from the symmetry being gauged. For U(1) we subtracted a new term $ieA_\mu$ but for a general symmetry with $n$ generators $\tau_1,\dots, \tau_n$, we will have to create $n$ new vector bosons and add define the covariant derivative as 
\begin{e}
  D_\mu = \del_\mu - ie\tau_1 A^1_\mu - \dots - ie\tau_n A^n_\mu.
\end{e}

Physically, the gauging of this symmetry creates a long-range force between fermions. If the phase of fermions were a global symmetry, then positrons and electrons would be identical except in their parity and time properties and would never interact. But electromagnetism arose when we gauged that symmetry. Every vector boson in the standard model is a consequence of a different symmetry being gauged, and all result in forces between fermions. 

Gauging a global symmetry is an excellent tool for simplifying the standard model. One needn't explain why vector bosons are there if a gauge symmetry \textit{requires} them to be there. It was also a crucial historical development, since seemingly bizarre interactions such as the weak force could be explained by proposing a global symmetry and gauging it. Later in this book, we will explain weak interactions by gauging isospin symmetry, and strong interactions by gauging color symmetry.

This connection between interactions and gauge symmetries was first proposed in 1915, well before the invention of QFT, by Emmy Noether. A mathematician by training and profession, Noether published her incredibly influential Noether's Theorem during a brief six-month break from math, though she was prevented from publishing the result for several years due to institutional sexism against her and her work. The theorem states that any physical theory a gauge symmetry contains a corresponding conserved charge (a charge for which she gives a formula). Our above statement connecting gauge symmetries to vector bosons is same statement in a QFT context, since vector bosons conserve charge.

\section{Feynman Rules and Conserved Quantities}

Before we compute any scattering amplitudes involving vector bosons we need to find the Feynman rules for the new particle. Vector bosons are typically denoted with wavy lines in diagrams, so that the $e\overline \psi\slashed A \psi$ interaction term is depicted as 
\begin{center}
  \feynmandiagram [horizontal=a to b, scale=0.8] {
    a -- [photon, particle=$\mu$] b,
    f1 [particle] -- [fermion] b -- [fermion] f2 [particle],
  };$=-ie\gamma^\mu.$
\end{center}
When we introduced the fermion, we used the arrows to distinguish between $\psi$ and $\overline \psi$. Now we can also interpret them as arrows pointing in the direction of the flow of charge, and the vector boson interacts with these charge arrows because it carries the charge force. Furthermore, the vector bosons don't have arrows on them but they do have spacetime indices. These indices are contracted with the $\gamma^\mu$ of a vertex when they meet fermions.

To find the propagator of the vector boson we first find the differential operator corresponding to the kinetic term:
need to compute the Green's function of the differential operator in the Lagrangian. The kinetic term in the Lagrangian can be written as
\begin{es}
  -\frac{1}{4}F^{\mu \nu}F_{\mu \nu}&= -\frac{1}{4}\parens{\del_\mu A_\nu - \del_\nu A_\mu}\parens{\del^\mu A^\nu - \del^\nu A^\mu}\\
  &=-\frac{1}{2}\parens{(\del_\mu A_\nu)^2 - \del_\nu A_\mu\del^\mu A^\nu}\\
  &=\frac{1}{2}A_\mu\parens{\del^2\delta^\mu_\nu  - \del_\nu\del^\mu}A^\nu\\
  &=\frac{1}{2}A_\mu\parens{\del^2\delta^\mu_\nu}A^\nu\\
\end{es}
where the third line comes from integration by parts, which we may use because this kinetic term is always integrated over in the action, $S = \int d^4 x \, \mathcal{L}$. The last line acknowledges that $\del_\mu \del^\nu \rightarrow -\del_\mu \del^\nu$ with a second integration by parts, and therefore this term will vanish in the action. Therefore, the differential operator for a vector boson is just the scalar field operator times $\delta^\mu_\nu$ and the Green's function follows the same pattern. It is convenient to write the propagator with both indices down for a vector boson photon, so
\begin{e}
  G_\mathrm{spin\ 1}(p^2) = -\frac{g_{\mu \nu}}{p^2 + i \epsilon}
\end{e}
is the propagator of a vector boson.

\subsection{Polarization}

Finally, there is the question of how a vector boson plane wave manifests itself in experiment. The value of $A^\mu$ for a plane wave of vector bosons is observable just as the value of a spinor was observable for a plane wave of fermions. For vector bosons, this value $\epsilon^\mu$ is called the polarization. 

However, different $\epsilon^\mu$ vectors might correspond to the same physical system. For example, the vector's normalization is not observable. To remove ambiguity, we therefore normalize polarization according to $\epsilon^\mu \epsilon_\mu = -1$.  Another ambiguity is caused by gauge invariance, which we could choose to remove by picking a gauge. For example, we already discussed that $\del_\mu A^\mu$ is not gauge-invariant, so if we set $\del_\mu A^\mu=0$ we have broken the invariance and picked a gauge known as the Lorentz gauge. If this derivative is computed for a plane wave, it states that $p_\mu \epsilon^\mu = 0$, meaning that polarization is \emphi{transverse}. \jtd{This isn't fully right yet}.

Thus, if a vector photon were traveling along the $z$-axis, the possible polarization vectors are superpositions of $\epsilon^\mu = (0, 1, 0, 0)$ and $\epsilon^\mu = (0, 0, 1, 0)$. To compute a Feynman diagram, the polarizations of the external legs should be multiplied onto the $g_{\mu\nu}$ terms in the vector propagators, just as spinors are multiplied onto the Dirac $\gamma^\mu$ matrices of fermion propagators. If the internal or external plane waves are unpolarized, we can instead multiply the diagram by $-g^{\mu \nu}$ where $\mu$ and $\nu$ are the indices corresponding to those external legs.


\section{Coulomb Potential}
\label{sec:vector-coulomb}

Let's address any ambiguity from the previous section by computing a Feynman diagram. Inspired by the near-reproduction of the Coulomb potential in the previous chapter, we'll study the same diagram, but replace the force-carrying scalar with a force-carrying gauge boson:
\begin{center}
  \feynmandiagram [horizontal=a to b, scale=0.8] {
    i1 -- [fermion,momentum'=$p'$] a -- [fermion,momentum=$p$] i2,
    a -- [photon,momentum=$p'-p$] b,
    i3 -- [fermion,momentum=$q$] b -- [fermion,momentum'=$q'$] i4,
  };
\end{center}
The contribution from this diagram is 
\begin{e}
  iM = \frac{1}{4}\tr\brackets{(-ie\gamma^\mu)\frac{-g_{\mu \nu}}{p^2 + i \epsilon} (-ie \gamma^\nu)}.
  \label{eqn:photon-coulomb-amplitude}
\end{e}
As you can see, the indices of the propagator are contracted with the vertex so that $M$ remains a scalar. The trace indicates that we assume the scattering beams do not have a coherent spin. Since the trace of $\gamma^\mu g_{\mu\nu} \gamma^\nu$ is 4,
\begin{e}
  iM = \frac{e^2}{p^2 + i \epsilon}.
\end{e}
This amplitude is identical to the Coulomb potential amplitude we computed for a scalar force-carrier except with the opposite sign, meaning that in the vector boson case positrons and electrons are attracted rather than repelled, with a Coulomb potential of 
\begin{e}
  V(r) = -\frac{e^2}{4\pi r}
\end{e}
If we thought of the vector boson as a photon and the fermion as the electron, this Coulomb potential is exactly that of electromagnetism! Furthermore, if the fermion beams had a coherent spin then the $\gamma^\mu$ and $\gamma_\nu$ matrices flip the spin of each particle, and the $g_{\mu \nu}$ term ties together the spin flips so that both occur simultaneously and equally across all three spatial axes. Thus, the ``photon'' can be thought of as transferring spin, inducing a total spin change of $1$ in both particles. This is another demonstration of why vector bosons are called spin one.

The fact that the vector boson interacts with spin cements our suspicion that a photon is a vector boson. Interaction with spin is the defining property of magnetic fields as we discuss in the next section, and we've already shown that photons interact with charge similarly to an electric field. Thus, vector bosons are true electromagnetic waves, and \ref{eqn:qed-lagrangian} can truly be thought of as the Lagrangian describing photons and electrons, and the electromagnetic force. In the context of quantum field theory, we call the theory Quantum Electrodynamics, or QED. 

\section{Magnetic Dipole Moment}
Another interesting interaction to consider is the scattering of an electron off of an ambient electric or magnetic field. The leading order diagram is simply
\begin{center}
  \feynmandiagram [horizontal=b to a, scale=0.8] {
    a -- [photon] b[particle=$\epsilon^\mu$],
    i1 -- [fermion] a -- [fermion] i2,
  };
\end{center}
which contributes the amplitudes
\begin{e}
  iM = -ie \epsilon_\mu\overline v\gamma^\mu u
  \label{eqn:leading-order-mag-mom}
\end{e}
where $u$ and $\overline v$ are the spinors of the incoming and outgoing electrons. \jtd{Where's the electric force?} This is reminiscent of the Hamiltonian for spin-magnetic field interaction in non-relativistic quantum mechanics:
\begin{e}
  H = -g\frac{e}{2m}\bm B \cdot \bm \sigma
\end{e}
The $g$-factor is a unitless constant measured by experiment. It has no official name, but the quantity $g \frac{e}{2m}$ is referred to as the \emphi{magnetic moment}.
where $e$ is the electron charge, $m$ is its mass, $\bm \sigma$ is the spin operator, and $\bm B$ is the magnetic field. Applying the rules of non-relativistic quantum mechanics to compute a scattering amplitude would remove the mass dependence, yielding
\begin{e}
  iM = -ie\frac{g}{2}\bm B \cdot \bm \sigma.
  \label{eqn:non-rel-mag-mom}
\end{e}

Though they use different notation, (\ref{eqn:leading-order-mag-mom}) and (\ref{eqn:non-rel-mag-mom}) are very similar. They have the same sign, are both proportional to $e$, and both depend on the dot product of the magnetic field (either $\bm B$ or $\epsilon^\mu$) with spin (either $\overline v\gamma^\mu u$ or $\bm \sigma$). Averaging over either spin or polarization would set both equal to zero. This leaves only one difference between (\ref{eqn:leading-order-mag-mom}) and (\ref{eqn:non-rel-mag-mom}), which is the presence of $g$. In fact, if $g=2$, then the QFT scattering amplitude would exactly equal the non-relativistic scattering amplitude.

When this $g=2$ conclusion was originally published by Paul Dirac, it represented the first ever correct theoretical determination of the magnetic moment of the electron. A classical model in which the electron is a spinning solid sphere with uniformly distributed charge would yield (incorrectly) $g=1$. Thus, the $g=2$ conclusion is a fundamental result of adding relativity to quantum mechanics.

If we pressed on to higher order Feynman diagrams, however, we would be forced to break modify this conclusion. The next order diagram is
\begin{center}
  \feynmandiagram [horizontal=b to a, scale=0.8] {
    a -- [photon] b,
    j1 -- [photon] j2,
    i1 -- [fermion] j1 -- [fermion] a -- [fermion] j2 -- [fermion] i2,
  };
\end{center}
which would both contribute a correction to $e$ and to $g$ in order to maintain the equivalence of (\ref{eqn:leading-order-mag-mom}) and (\ref{eqn:non-rel-mag-mom}). These corrections both have important effects. The change in $e$ is difficult to measure unless an experiment is conducted in a very small space, where this second order diagram can be suppressed, in which case the uncorrected value of $e$ is observed instead. This effect is responsible for the Lamb shift in the spectrum of hydrogen, where the electron is extremely close to the nucleus.

The correction to $g$ is easier to measure since the uncorrected value $g=2$ is known already. QED's prediction for the $g-2$ correction is a great achievement of QED and is confirmed by experiment to many decimal places. In the next chapter, we develop the mathematical tools necessary to make this prediction.

In the next section we'll discuss a somewhat different but equally important historical achievement of QED: the prediction for how relativistic electrons scatter off matter. This ``Compton scattering'' cross section is of great importance in astrophysics.

\section{Compton Scattering}

\section{The Ward Identity \& Massless Photons are Protected}

\begin{problem}[Repulsion between Electrons]
  Use the same techniques as section \ref{sec:vector-coulomb} to compute the Coulomb potential between two like-charged particles. Conclude that these particles repel.
\end{problem}