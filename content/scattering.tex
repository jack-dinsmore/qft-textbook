\chapter{Scattering}
\label{chap:scattering}

As defined in chapter \ref{chap:intro}, the quantum state $\hat \phi(x)\ket{0}$ represents a $\phi$ particle created at spacetime position $x$. The Heisenberg uncertainty principle requires that this particle therefore have no definite momentum. If we wished, we could switch to the momentum basis by defining a new operator 
\begin{e}
  \hat \phi(k) = \int d^4 x \, e^{-ik\cdot x} \hat \phi(x)
\end{e}
which is the Fourier transform of $\hat \phi(x)$. Note that $k \cdot x$ is the dot product of four-vectors, or $k^0 x^0 - \bm k \cdot \bm x$. Under this definition, $\hat \phi(k)\ket{0}$ represents a particle with fixed momentum $k$, and this time with indefinite position.

This plane-wave configuration corresponds physically to a long beam of particles of fixed momentum, such as a laser. These beams are the incoming and outgoing states of a scattering experiment, which is the natural tool used to understand the quantum nature of particles. We will therefore use the momentum basis for most of this chapter.

\section{\texorpdfstring{$S$}{S} Matrix and Scattering Amplitudes}
Suppose a scattering experiment's incoming beam contains $n$ particles with four-momentum $p_1$ $p_2$, $p_3$, $\dots, p_n$. The above notation allows us to write this incoming state as 
\begin{e}
  \ket{\mathrm{in}} = \hat \phi(p_1)\cdots\hat \phi(p_n)\ket{0}.
\end{e}
and likewise we can name the outgoing state
\begin{e}
  \ket{\mathrm{out}} = \hat \phi(q_1)\cdots\hat \phi(q_m)\ket{0}.
\end{e}
Note that the number of incoming particles $n$ need not equal the number of outgoing particles because in a relativistic theory, particles can be converted to energy unlike in non-relativistic quantum mechanics.

The overlap between the input and output states is usually called the $S$ matrix, defined with an extra factor of $i$ due to convention.
\begin{e}
  iS = \braket{\mathrm{out}|\mathrm{in}} = \braket{0|\phi(q_m)^\dagger\cdots\hat \phi(q_1)^\dagger\hat \phi(p_1)\cdots\hat \phi(p_n)|0}.
\end{e}
The magnitude of the $S$ matrix is an observable quantity because the probability of a scattering event with these initial and final momenta is $|S|^2$. The right hand side is also important because it is an $(n+m)$PCF, which the QFT principle of least action tells us how to compute.

If no interactions occur, we expect no momentum transfer, so $n=m$ and $p_j = q_j$ for all $j$. In other words, $iS \propto \prod_i \delta(p_j - q_j)$. Even when interactions are possible, many particles will still not interact and will pass through the beam unchanged. We therefore define a new quantity $T$ such that
\begin{e}
  iS_{p \rightarrow q} = \prod_i \delta(p_j - q_j) + iT_{p\rightarrow q}
\end{e}
where the $T$-matrix is the ``off-diagonal'' part of the $S$-matrix, which quantifies the scattering interactions.

Even when interactions are present, $\sum_i p_i = \sum_j q_j$ always holds due to momentum and energy conservation. To extract this fact we make one more definition:
\begin{e}
  iT_{p \rightarrow q} = (2\pi)^4 \delta\parens{\sum_i p_i - \sum_j q_j}iM_{p\rightarrow q}
\end{e}
where $M$ is called the \emphi{scattering amplitude}\footnote{It's worth noting that $M$ is Lorentz invariant because the original definition of the $S$ matrix is invariant as long as the states $\ket{\mathrm{in}}$ and $\ket{\mathrm{out}}$ are invariant, which is ensured as long as the states are normalized correctly. The factors we subtracted off the $S$ matrix to form $M$ are also Lorentz invariant because they are delta functions of four-vectors}. $M$ is the interesting physical quantity behind scattering experiments; its value is set by the fundamental physics rather than mere conservation laws or unscattered particles. Most of the rest of this book will therefore compute the scattering amplitude and stop there.

There are some additional complexities which we have glossed over but will address later. One is that momentum is not the other measurable property of a particle. Other conserved quantities, such as fermion spin or photon polarization, can be observed and should therefore be included in the initial and final states. We neglected these additional quantities for the sake of clarity and will add them back in when fermions and photons are discussed.

The reader may also be curious about the properties of this $\hat \phi$ operator. It's definition and its action against $\ket{0}$ have been given, but other details such as the commutator $[\hat \phi(x), \hat \phi(y)]$ and the structure of the Hilbert space it belongs to have been omitted. This is because we will rarely use the $\hat \phi$ operators themselves. They were useful to derive the QFT principle of least action and to understand why $|S|^2$ is the probability of a scattering experiment, but now we can forget about the operators and think in terms of the $S$ matrix and scattering experiments. They will only appear again when we discuss Fermi's golden rule and second quantization in section \ref{sec:second-quantization}.


\section{Decay Widths and Cross Sections}

It's worth making the connection between scattering amplitudes and experiment more explicit. The connection is quite simple except for the fact that scattering amplitudes are Lorentz invariant and experiments, which are done at fixed velocities and orientations, are not. Scattering amplitudes and the outcome of scattering experiments can therefore differ by a energy-dependent coefficient, which we will work out by normalizing the incoming and outgoing plane waves.

Let's consider a single particle of mass $m$ scattering into $n$ final particles, with momentum $q_j$. A single-particle scattering event is also known as particle decay, and the measurable quantity is the decay width $\Gamma$ which is the probability of scattering per unit time in the rest frame of the particle. The lifetime $\tau$ of the particle is related to $\Gamma$ by $\tau = 1/\Gamma$.

Converting the momentum-space fields $\hat \phi(p)$ back to position-space via the inverse Fourier transform gives us
\begin{e}
  \hat \phi(x) = \int \frac{d^4 k}{(2\pi)^4}\, e^{-ik \cdot x}\delta(k^2 - m^2) \hat \phi(k).
\end{e}
The delta function can be inserted with no penalty because of the relativistic constraint that $k^2 = m^2$, where $m$ is the particle mass. This constraint is named the \emphi{on-shell constraint}, so called because the surface $k^2 = m^2$ looks like a hyperboloid of two sheets in $k$-space. That is, a ``shell''.

If we limit our focus to the incoming and outgoing beams, we can replace $\hat \phi(x)$ with the wavefunction $\psi(\bm x)$. This is because we already understand the state $\hat \phi(x)\ket{0}$ --- the single particle state --- in non-relativistic quantum mechanics by describing its wavefunction as a plane wave. During a scattering experiment, we cannot rely on plane wave knowledge from non-relativistic quantum mechanics and must instead rely on the principle of least action, which converts the operators to a path integral of functions $\phi$ rather than a single wavefunction. 

Performing this replacement,
\begin{e}
  \psi(t, \bm x) = \int \frac{d^4 k}{(2\pi)^4}\, e^{i\bm k \cdot \bm x}e^{-i k^0 t}\delta(k^2 - m^2) \psi(\bm k).
\end{e}
Since experiments measure the spatial part of momentum, we'll integrate out the $k_0$ component. The fact that
\begin{e}
  \int dx\, \delta(f(x)-f(z)) = \frac{2\pi}{|f'(z)|}\delta(x-z),
\end{e}
leads to
\begin{e}
  \psi(\bm x) = \int \frac{d^3 \bm k}{(2\pi)^3}\, e^{i\bm k \cdot \bm x} \frac{e^{-iE_{\bm k} t}}{\sqrt{2E_{\bm k}}}\psi(\bm k)
\end{e}
where $E_{\bm k} = k_0 = \sqrt{m^2 + \bm{k}^2}$ is the energy of the particle at momentum $\bm k$.

Wavefunctions must be normalized, meaning
\begin{e}
  1 = \int d^3\bm x\, |\psi(\bm x)|^2 = \int \frac{d^3 k}{(2\pi)^3}\, \frac{1}{2E_{\bm k}}|\psi(\bm k)|^2
\end{e}
The lesson to take from this is that the observed wavefunction $|\psi(\bm k)|^2$ differs from Lorentz invariant operators $\hat \phi(k)$ by a factor of $1/2E_{\bm k}$. Thus, the probability to decay into the exact state of $q_1,\dots, q_m$ per unit time is
\begin{e}
  d\Gamma = \brackets{\frac{1}{2 E_p}}\brackets{\prod_j\frac{d^3 q_j}{(2\pi)^3}\frac{1}{2 E_{q_j}}} \brackets{|M_{p\rightarrow q}|^2(2\pi)^4 \delta\parens{p - \sum_j q_j}}
  \label{eqn:diff-decay-rate}
\end{e}

The full decay rate integrates over all possible final states to give the true lifetime of the particle:
\begin{es}
  \Gamma &= \int d\Gamma\\
  &= \frac{1}{2 E_p}\int \frac{d^3 q_1}{(2\pi)^3 (2E_{q_1})}\dots\int \frac{d^3 q_m}{(2\pi)^3 (2E_{q_m})}|M_{p\rightarrow q}|^2 \delta\parens{p - \sum_j q_j}
  \label{eqn:total-decay-rate}
\end{es}

Similar techniques can be used to find the probability of interaction for multiple to multiple particle scattering; in this case, the probability is called a cross section $\sigma$. The differential cross section $d\sigma$ is very similar to the equation for $d\Gamma$, except that there may be additional coefficients due to the fact that multiple momenta $\bm p_i$ are now unknown. For example, for two incoming particles with velocity $v_1$ and $v_2$, the differential cross section is
\begin{es}
  d\sigma = &\frac{1}{|\bm v_1 - \bm v_2|}\parens{\prod_i\frac{d^3 p_i}{(2\pi)^3}\frac{1}{2 E_{\bm p_i}}}\parens{\prod_j\frac{d^3 q_j}{(2\pi)^3}\frac{1}{2 E_{\bm q_j}}}\\
  &\times|M_{p\rightarrow q}|^2(2\pi)^4 \delta\parens{\sum_u p_i - \sum_j q_j}.
\end{es}

These cases of 1 and 2 particles cover most scattering events. To evaluate the total probability of scattering, $d\Gamma$ or $d\sigma$ may be integrated over all momenta. If identical particles $A$ and $B$ are present in the initial and final states, the states where $A$ and $B$ are switched should count only once. Thus, for $k$ types of $m_j$ identical particles, we should divide the cross section by $\prod_{j=1}^k m_j!$.