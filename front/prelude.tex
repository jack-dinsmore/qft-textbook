\chapter{Abstract}

Quantum Field Theory, or QFT, is our most fundamental description of the universe. It combines the lessons of quantum mechanics with the laws of special relativity to create a theory consistent with both. Except for gravity, quantum field theory predicts every phenomenon yet observed, with only twenty or so free parameters to be determined by experiment. Below, we provide a \jtd{number} page summary of QFT, with links to the chapters in which each topic is described.

The need for QFT arose from a variety of sources. Theorists were bothered by the fact that quantum mechanics is inconsistent with relativity (for example, the \Schrodinger equation is not Lorentz invariant), while relativity is inconsistent with quantum mechanics. Experimentally, new particles such as the positron were starting to appear and a theory that could create, destroy, and propagate these particles was necessary. With the benefit of hindsight, we present in chapter \ref{chap:intro} a principle of least action and argue that it is consistent with relativity and quantum mechanics.

The left hand side of this principle of least action is an $n$ point correlation functions of particles ($n$PCFs), where $n$ is a positive integer. The right hand side is an integral over every path the $n$ particles could take of $e^{iS}$, where $S$ is the action. An $n$PCF may seem like an esoteric object at first, but we show that it is directly connected to scattering (chapter \ref{chap:scattering}). For example, the 4PCF encodes the probability for two particles to scatter off each other. The 2PCF is special; it encodes many of the characteristic properties of a particle, including its mass, its ability to scatter with and decay into other particles, and even the energies of its bound states (chapter \ref{chap:spectrum}).

Now that the $n$PCFs are understood, we tackle the task of solving for an $n$PCF using the QFT principle of least action. The result depends on the action $S$ used, so we suggest a simple action for a particle with no spin, called a scalar particle (chapter \ref{chap:spin-zero}). We first solve for the $n$PCFs in a theory without interactions. This can be done exactly, and using our understanding of how $n$PCFs are related to scattering, we confirm some of the properties we expect a relativistic theory such as QFT to have. (For example, we confirm that the theory is causal.)

Next, we solve for the $n$PCFs when the scalar particles are allowed to interact with each other. This cannot be done exactly, so we use perturbation theory. The mathematics of this process are long, but are greatly simplified by the use of Feynman diagrams. After a few calculations, we are able to compute our first scattering amplitudes for an interacting theory. This process leads to a remarkable intuition which holds for any weakly interacting theory: all interactions come about by the creation of virtual particles, which are like a temporary excited quantum state used by the interacting particles to tunnel into another state. Just as an electron can tunnel through an energy barrier in non-relativistic quantum mechanics, now a relativistic particle uses the relativistic equation $E=mc^2$ to create virtual particles as a means of tunneling through an energy barrier.

We also show that a light scalar particle is able to carry information about the presence of other particles over long distances. To an observer who cannot see the light scalar particle, this may look like a long-range force. For a light enough scalar, we derive that this force is proportional to $1/r^2$, like Coulomb's force. However, we also suggest that it is unlikely that scalar particles can carry forces like this in Nature as it theories which have light enough scalars are rare.

Having treated the simplest case of a scalar particle, we consider particles with spin one-half (chapter \ref{chap:spin-one-half}), or fermion. Immediately, we notice that these fermions must behave differently from scalars because of the Pauli exclusion principle, which says that two fermions cannot exist in the same quantum state. Fortunately, the spin statistics theorem allows us to encode this Pauli exclusion principle in the action, so we can derive $n$PCFs for fermions using the same techniques as with scalars while remaining consistent with the Pauli exclusion principle.

With these $n$PCFs, we propose a new model that binds the neutron to the proton in a nucleus.The model treats protons and neutrons as fermions and proposes a light scalar known as a pion to transfer force between them. We compute several predictions of this model which are consistent with experiment to first order, though experiments also tell us that protons, neutrons, and pions have substructure which this model does not contain. Emboldened, we suggest a similar model for the Coulomb force between a positron and an electron. However, this model cannot explain magnetic fields, so we reject it. \jtd{Symmetry breaking}

In order to pursue magnetic fields, we further consider spin-one particles (chapter \ref{chap:spin-one}). Like scalars, these particles are bosons so they do not obey the Pauli exclusion principle and can carry forces. However, a spin-one particle has too many degrees of freedom to be detectable in four dimensions. We propose the removal of one of these degrees of freedom by suggesting that the spin-one particle obeys some self-symmetry, in that a certain transformation of the spin-one field leaves the action invariant. This removes one physical degree of freedom from the particle, making it observable. It also limits the number of possible actions down to a very small number. We choose the simplest action, allow the spin-one particle to interact with the electron, and we see that the spin-one particle has all the properties of a photon. This spin one-half electron plus a spin-one photon theory is named Quantum Electrodynamics, or QED. We spend the rest of the chapter showing that it reproduces magnetism and predicts that an electron has a magnetic dipole moment of about two, consistent with experiment. We also show that this theory guarantees the photon to be massless.

We would like to compute the predictions of QED to see how well they hold. We start with electron to electron scattering, but \jtd{infrared}. We move on to compute the magnetic moment of the electron $g$, but again we find an infinite prediction. We interpret this infinity as an unphysical prediction coming from the fact that the virtual particles that the electron uses to tunnel between states can be very high energy. We don't necessarily expect QFT to be valid at very high energies where we have not tested the theory. We therefore invent a technique known as renormalization to subtract off these high-energy predictions, leaving us only with the predictions of lower-energy QFT. Using this technique, we show that QED agrees exactly with experiment in the magnetic moment of the electron, the spectrum of hydrogen, and the probability for electrons and positrons to annihilate. We also show a few additional calculations which reveal new properties of electrons and photons which non-relativistic quantum mechanics was ignorant of.

\jtd{Non-abelian}